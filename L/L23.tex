\section{Lecture 23 - 12.30}

\begin{recap}[Review]
\begin{align*}
    L[y] &= -\left( a(t) \frac{d^2 y}{dt^2} + b(t) \frac{dy}{dt} + c(t) y \right) = \lambda r(t) y \ \alpha < t < \beta \ (r \equiv 1) \\
    B_\alpha[y] &= a_1 y(\alpha) + b_1 \frac{dy}{dt}(\alpha) = 0 \ |a_1| + |b_1| > 0 \\
    B_\beta[y] &= a_2 y(\beta) + b_2 \frac{dy}{dt}(\beta) = 0 \ |a_2| + |b_2| > 0
\end{align*}
\end{recap}

\(\lambda\) is an eigenvalue of \(L\) with bdry. conditions \(B_\alpha[y] = B_\beta[y]\) iff there exists a nontrivial solution \(y\) to \(L[y] = \lambda r y\), \(B_\alpha[y] = B_\beta[y] = 0\). The function \(r\) is called a weight.

If \(a \in C^2[\alpha, \beta]\) and \(b \in C^1[\alpha, \beta]\) then
we say that \(L\) is self-adjoint
\begin{align*}
    \iff \left( L[y], z \right) = \left( y, L[z] \right) \ \forall y, z \in V = C^2[\alpha, \beta], B_\alpha[y] &= B_\beta[y] = 0 \\
    B_\alpha[z] &= B_\beta[z] = 0
\end{align*}
iff \(b = \frac{da}{dt}\) in \([\alpha, \beta]\), where \((u, v) = \int_\alpha^\beta u v \, dt\).

If \(L\) is self-adjoint and we denote \(p = a\) (so \(b = \frac{dp}{dt}\)) and \(q = -c\), then
\[L[y] = -\left( \frac{d}{dt} \left( p(t) \frac{dy}{dt} \right) - q(t) y \right) \rightsquigarrow \text{Sturm-Liouville operator}\]

\begin{example}
\[\begin{cases} \frac{d^2 y}{dt^2} + \lambda y = 0 & 0 < t < L \\ y(0) = y(L) = 0 \end{cases}, \ L[y] = -\frac{d^2 y}{dt^2}\]

\(\lambda_n = \left(\frac{\pi n}{L}\right)^2\) for \(n=1, 2, \cdots\) are the only real eigenvalues.

\(y_n(t) = \sin\left(\frac{\pi n}{L} t\right), \ t \in [0, L]\), are the corresponding eigenfunctions.

It can be shown that if \(f \in C^1[0, L]\) with \(f(0)=f(L)=0\), then \(\underbrace{\sum_{n=1}^\infty (f, y_n) y_n}_{\text{eigenfunction expansion of } f}\) converges uniformly to \(f\) in \([0, L]\).
\end{example}

\begin{definition}[Regular Sturm-Liouville problem]
    The problem
    \[\begin{cases}
        \frac{d}{dt} \left( p(t) \frac{dy}{dt} \right) - q(t) y = -\lambda r y \ \alpha < t < \beta \\
        B_\alpha[y] = B_\beta[y] = 0
    \end{cases}\]
    with \(|a_1| + |b_1| > 0\), \(|a_2| + |b_2| > 0\), \((\alpha, \beta)\) finite, \(p, r, \frac{dp}{dt}, q \in C[\alpha, \beta]\), \(p(t) > 0, r(t) > 0\) for every \(t \in [\alpha, \beta]\), is called a \textbf{regular Sturm-Liouville problem}.
\end{definition}

\begin{theorem}
Consider a regular Sturm-Liouville problem as in the previous definition. Then,
\begin{enumerate}[a)]
    \item Eigenvalues exist and all of them are real.
    \item Eigenvalues are simple, i.e. the associated eigenspaces have dimension 1.
    \item Eigenfunctions are real, apart from possible complex multiplicative constants. Also, eigenfunctions with different eigenvalues are orthogonal, i.e. \((y, z) = 0\) if \(y, z\) are eigenfunctions with different eigenvalues, and \((y, z) = \int_\alpha^\beta r y z \, dt\).
    \item There are countable infinite eigenvalues, \(\lambda_0, \lambda_1, \cdots\), and they can be ordered as \(\lambda_0 < \lambda_1 < \lambda_2 < \cdots\) and the index \(n\) is the number of zeros of the eigenfunction \(y_n\) with eigenvalue \(\lambda_n\), in \((\alpha, \beta)\). Also, the zeros of \(y_{n+1}\) interlace the zeros of \(y_n\), and \(\lim_{n \to \infty} \lambda_n = +\infty\).
    \item If \(f \in C^1[\alpha, \beta]\) then the eigenfunction expansion of \(f\), \(\sum_{n=0}^\infty B_n y_n\) converges uniformly to \(f\) in any closed interval \(I \subset (\alpha, \beta)\), where \(B_n = \frac{(f, y_n)}{\|y_n\|}\), \(n=0, 1, 2, \cdots\).
\end{enumerate}
\end{theorem}

\begin{proof}
    For simplicity, we assume \(r \equiv 1\).
    
    \begin{enumerate}[a)]
        \item We accept the proof of the existence of eigenvalues without proof.
        
        If you are interested, see Codington-Levinson, Ch. 7.
    
        Let's prove that all the eigenvalues are real. First, we'll show that
        \[(L[y], z)_c = (y, L[z])_c\]
        for every \(y, z \in C^2[\alpha, \beta]\), possibly complex-valued, that satisfy
        \[B_\alpha[y] = B_\beta[y] = 0, \ B_\alpha[z] = B_\beta[z] = 0,\]
        where \((u, w)_c = \int_\alpha^\beta u \bar{w} \, dt\). Let \(y, z\) be two of these functions.
        \begin{align*}
            (L[y], z)_c &= \int_\alpha^\beta \left( \frac{d}{dt} \left( p(t) \frac{dy}{dt} \right) - q(t) y \right) \bar{z} \, dt \\
            &= \int_\alpha^\beta \frac{d}{dt} \left( p(t) \frac{dy}{dt} \right) \bar{z} \, dt - \int_\alpha^\beta q(t) y \bar{z} \, dt    
        \end{align*}

        Integration by parts in the first and middle term:
        \begin{align*}
            &= \left. p(t) \frac{dy}{dt} \bar{z} \right|_\alpha^\beta - \int_\alpha^\beta p(t) \frac{dy}{dt} \frac{d\bar{z}}{dt} \, dt - \int_\alpha^\beta q(t) y \bar{z} \, dt \\
            &= \left. p(t) \left( \frac{dy}{dt} \bar{z} - y \frac{d\bar{z}}{dt} \right) \right|_\alpha^\beta + \int_\alpha^\beta \left[ \frac{d}{dt} \left( p(t) \frac{d\bar{z}}{dt} \right) - q(t) \bar{z} \right] y \, dt \\
            &= \underbrace{\left. p(t) \left( \frac{dy}{dt} \bar{z} - y \frac{d\bar{z}}{dt} \right) \right|_\alpha^\beta}_{\substack{=0 \text{ (see proof of } b=a') \\ B_\alpha[\bar{z}]=B_\beta[\bar{z}]=0 \text{ (linear and with real coeff.)}}} + \int_\alpha^\beta \underbrace{\left[ \frac{d}{dt} \left( p(t) \frac{d\bar{z}}{dt} \right) - q(t) \bar{z} \right]}_{\substack{-L[\bar{z}] = -\overline{L[z]} \\ \text{ (linear and with real coeff.)}}} y \, dt \\
            &= -\int_\alpha^\beta y \overline{L[z]} \, dt = -(y, L[z])_c
        \end{align*}

        So \((L[y], z)_c = (y, L[z])_c\).

        Assume that \(\lambda \in \mathbb{C}\) is an eigenvalue with eigenfunction \(y\). Then,
        \[\underbrace{(L[y], y)_c}_{\lambda (y, y)_c} = (y, \underbrace{L[y]}_{\lambda y})_c = \int_\alpha^\beta y \overline{\lambda y} \, dt = \bar{\lambda} (y, y)_c\]
        
        That is, \(\lambda (y, y)_c = \bar{\lambda} (y, y)_c\).
        
        As \(y\) is an eigenfunction, \(y \not\equiv 0\), we deduce that \(\lambda = \bar{\lambda}\), so \(\lambda \in \mathbb{R}\).

        \begin{remark}
            \(z\) satisfies \(B_\alpha[z] = a_1 z(\alpha) + b_1 \frac{dz}{dt}(\alpha) = 0\). Then,
            \[0 = \overline{B_\alpha[z]} = \overline{a_1} \overline{z(\alpha)} + \overline{b_1} \overline{\frac{dz}{dt}(\alpha)} \underset{\substack{\uparrow \\ a_1, b_1 \in \mathbb{R}}}{=} a_1 \bar{z}(\alpha) + b_1 \frac{d\bar{z}}{dt}(\alpha) = B_\alpha[\bar{z}]\]
        \end{remark}

        \item Let \(\lambda\) be an eigenvalue with eigenfunctions \(y, z\). We need to prove that \(y\) and \(z\) are linearly dependent. Recall that \(y, z\) solve
        \[p(t) \frac{d^2 y}{dt^2} + \frac{dp(t)}{dt} \frac{dy}{dt} - q(t) y = \lambda y, \ \alpha < t < \beta\]
        
        As \(y, z\) are solutions to a linear second-order eq with continuous coeff, we know that \(W(y, z)\) is identically zero or never vanishes. Also, the first case tells us that \(y, z\) are linearly dependent.
        \[W(y, z)(t) = \det \begin{pmatrix} y(t) & z(t) \\ \frac{dy}{dt}(t) & \frac{dz}{dt}(t) \end{pmatrix} = y(t) \frac{dz}{dt}(t) - \frac{dy}{dt}(t) z(t)\]
    
        We've proved in the theorem of characterization of self-adjoint operators that
        \[p(t) \left( z \frac{dy}{dt} - y \frac{dz}{dt} \right) \Big|_\alpha^\beta = 0\]
        if \(y, z\) satisfy the bdry conditions.
        
        As this is our case, \(p(t) W(y, z)(\alpha) = 0\). (since \(p(\alpha) > 0\), boundary term logic implies Wronskian at boundary is zero).
        Then \(y, z\) are linearly dependent.
    \end{enumerate}
\end{proof}
