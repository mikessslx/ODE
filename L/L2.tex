\section{Lecture 2 - 10.16}

\begin{recap}
\begin{itemize}
\item The general form of an ODE is \(F(t, y, y', \cdots, y^{(n)}) = 0\) (\(*\)). A solution \(y\) to eq. (\(*\)) is a function that satisfies \(F(t, y(t), y'(t), \cdots, y^{(n)}(t)) = 0\) for every \(t\) in some interval.
\item \sout{Classification of ODEs:}
\begin{itemize}
\item \sout{Order}
\item \sout{ODE (scalar), Systems of ODEs}
\item \sout{Linear and nonlinear eqs.}
\end{itemize}
\item \sout{Well-posed problems:}
\begin{itemize}
\item \sout{Existence of solution}
\item \sout{Uniqueness of solution}
\item \sout{Stability (or continuous dependence on data)}
\item \sout{Ex: \(\begin{cases} y' = f(t, y) \\ y(t_0) = y_0 \end{cases}\)}
\end{itemize}
\end{itemize}
\end{recap}

\subsection{Some Solution Methods for First-Order Equations}

\begin{definition}[First-order ODE]
The general form of a first-order equation is \(\frac{dy}{dt} = f(t, y)\).
\end{definition}

\subsection{Linear equations: The integrating factor method}

\begin{definition}[Linear First-Order ODE]
{\color{blue}The general form for a first-order linear ODE is
\[\frac{dy}{dt} + a(t)y = b(t) \tag{6}\]
(Here, \(f(t, y) = b(t) - a(t)y\)). Assume that \(a\) and \(b\) are continuous functions in \(\mathbb{R}\).}
\end{definition}

If we can find an equivalent equation of equation (6), in the form
\[\frac{d\tilde{y}}{dt} = \tilde{b}(t) \tag{7}\]
we are done, because we can solve (7) by direct integration. The idea is to look first for a function \(\mu\) such that
\[\mu(t)\left( \frac{dy}{dt} + a(t)y \right) = \mu(t)b(t) \tag{8}\]
(where \(\mu(t)\frac{dy}{dt} + \mu(t)a(t)y\) can be easily transformed into the form (7)).

\subsubsection{Core of the Integrating Factor Method}
The core of the integrating factor method is the following observation
\[\frac{d}{dt}\left( \mu(t) y \right) = \frac{d\mu}{dt} y + \mu(t) \frac{dy}{dt}.\]

Then, if \(\mu\) satisfies
\[\frac{d\mu}{dt} = a(t)\mu,\]
then equation (8) reduces to
\[\frac{d\left( \mu(t) y \right)}{dt} = \mu(t) b(t).\]

Additionally, we can find \(\mu\) by solving \(\frac{d\mu}{dt} = a(t)\mu\).

\subsubsection{Derivation of the Integrating Factor \(\mu(t)\)}
Assume that the equation \(\frac{d\mu}{dt} = a(t)\mu\) admits a solution \(\mu\) which is positive, that is, \(\mu(t) > 0\) for every \(t\). Then,
\[\frac{1}{\mu} \frac{d\mu}{dt} = a(t) \text{, that is } \ \frac{\mu'(t)}{\mu(t)} = a(t)\]
(where \(\frac{\mu'(t)}{\mu(t)} = \frac{d}{dt} \ln|\mu(t)|\)).

Integrating both sides, we get
\[\int \frac{d}{dt} \ln|\mu(t)| \, dt = \int a(t) \, dt\]
which simplifies to
\[\ln|\mu(t)| = \int a(t) \, dt + C_0\]
\[|\mu(t)| = e^{C_0} e^{\int a(t) \, dt} = C e^{\int a(t) \, dt}\]

Since we assumed that \(\mu\) is positive, we get
\[\mu(t) = C e^{\int a(t) \, dt} \ (C > 0).\]

{\color{blue}For simplicity, we set \(C = 1\), so the integrating factor \(\mu(t) = e^{\int a(t) dt}\).}

It’s easy to verify that \(\mu\) is a solution to \(\frac{d\mu}{dt} = a(t)\mu\).

Then, equation (8) can be written as
\[\frac{d}{dt} \left( e^{\int a(t) dt} y \right) = e^{\int a(t) dt} b(t)\]

Integration with respect to \(t\), yields
\[e^{\int a(t) dt} y = \int e^{\int a(t) dt} b(t) dt + C \ (C \in \mathbb{R})\]

{\color{blue}Therefore,
\[y(t) = e^{-\int a(t) \, dt} \left( \int b(t) e^{\int a(t) \, dt} \, dt + C \right), \ t \in \mathbb{R} \tag{10}\]}

\subsubsection{Summary on Integrating Factor and General Solution}

\begin{theorem}[General Solution to Linear First-Order ODE]
{\color{blue}In summary, we deduced that if eq. (6) has a solution, then the solution must be a function in the form (10).
    
Also, it's easy to verify that any function like (10) solves eq. (6) in \(\mathbb{R}\). So, formula (10) is called the general solution to eq. (6), \(\frac{dy}{dt} + a(t)y = b(t)\).}
\end{theorem}

\begin{definition}[Integrating Factor]
{\color{blue}The function \(\mu(t) = e^{\int a(t) dt}\) is called an integrating factor for eq. (6).}
\end{definition}
