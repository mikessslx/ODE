\section{Lecture 10 - 11.13}

\begin{recap}
\begin{itemize}
\item \sout{\(\frac{dy}{dt} = A(t)y + b(t)\), \(b = 0\).}
        
\sout{The set of solutions is a vector space of dimension \(n\).}
\begin{itemize}
\item \sout{\(\mathcal{B} = \{y^{(1)}, y^{(2)}, \cdots, y^{(n)}\}\);}
\item \sout{\(y_H = c_1 y^{(1)} + c_2 y^{(2)} + \cdots + c_n y^{(n)}\), \(c_1, \cdots, c_n \in \mathbb{R} \rightsquigarrow\) general solution to \(\frac{dy}{dt} = A(t)y\).}
\end{itemize}
        
\sout{\(\mathcal{B}\) is a basis of solutions.}

\item \sout{\(\frac{dy}{dt} = A(t)y + b(t)\), \(b \not\equiv 0\).}
\begin{itemize}
\item \sout{\(\mathcal{B}, y_H\) as before;}
\item \sout{\(y = y_H + y_p\), where \(y_H\) is the general solution to the associated homogeneous system \(\frac{dy}{dt} = A(t)y\), and \(y_p\) is a particular solution to \(\frac{dy}{dt} = A(t)y + b(t)\).}
\end{itemize}
\end{itemize}
\end{recap}

\begin{example}
\leavevmode
\begin{itemize}
\item \sout{\(\frac{dy}{dt} = y\), \(n=1\),}
\[y(t) = C e^t, \ y_1(t) = 2e^t, \ y_2(t) = -\frac{9}{11} e^t.\]
\item \sout{\(\frac{dy}{dt} = y, \ y(0) = 1\).}
\end{itemize}
\end{example}

\begin{remark}
\sout{\(\frac{dy}{dt} = \frac{dy_H}{dt} + \frac{dy_p}{dt} = A(t)y_H + A(t)y_p + b(t) = A(t)(y_H + y_p) + b(t) = A(t)y + b(t)\).}
\end{remark}

\subsection{Fundamental Matrix Solution}

Assume that \(\mathcal{B} = \{y^{(1)}, \cdots, y^{(n)}\}\) is a basis of solutions to \(\frac{dy}{dt} = A(t)y\).

{\color{blue}\begin{definition}[Fundamental Matrix Solution]
The matrix \(Q\) whose \(j\)-th column is \(y^{(j)}\) is called a fundamental matrix solution to \(\frac{dy}{dt} = A(t)y\).
    
In other words, if \(y^{(1)} = \begin{pmatrix} y_1^{(1)} \\ \vdots \\ y_n^{(1)} \end{pmatrix}, \cdots, y^{(n)} = \begin{pmatrix} y_1^{(n)} \\ \vdots \\ y_n^{(n)} \end{pmatrix}\), then
\[Q = \begin{pmatrix} 
| & & | \\
y^{(1)} & \cdots & y^{(n)} \\
| & & | 
\end{pmatrix} 
= \begin{pmatrix} 
y_1^{(1)} & \cdots & y_1^{(n)} \\ 
\vdots & & \vdots \\ 
y_n^{(1)} & \cdots & y_n^{(n)} 
\end{pmatrix}
\rightsquigarrow \text{Fundamental Matrix Solution.}\]
\end{definition}}

Observe that \(\frac{dQ}{dt} = A(t)Q\). Also, if \(y\) is a solution to \(\frac{dy}{dt} = A(t)y\), then there exists \(C \in \mathbb{R}^n\) s.t. \(y = QC\) (that is, \(y(t) = Q(t)C\) for every \(t\)).

Notice that: \(\det Q(t) \neq 0\) for every \(t \iff \det Q(t_0) \neq 0\) at some \(t_0\).

\subsection{Variation of Parameters}

Assume that \(a_{ij}\) and \(b_j\) are continuous functions in some open interval \(I\), for \(i, j = 1, \cdots, n\). Let \(\mathcal{B} = \{y^{(1)}, \cdots, y^{(n)}\}\) be a basis of solutions to \(\frac{dy}{dt} = A(t)y\).

{\color{blue}The function \(y(t) = C_1(t)y^{(1)}(t) + \cdots + C_n(t)y^{(n)}(t)\) is a solution to \(\frac{dy}{dt} = A(t)y + b(t)\) if \(C_1, \cdots, C_n\) are continuously differentiable functions in \(I\), such that \(Q(t) \frac{dC}{dt} = b(t)\) where \(Q\) is the fundamental matrix.}

Notice that \(y(t) = Q(t)C(t)\), \(t \in I\). Then,
\begin{align*}
\frac{dy(t)}{dt} &= \underbrace{\frac{dQ(t)}{dt}}_{A(t)Q(t)} C(t) + Q(t) \underbrace{\frac{dC(t)}{dt}}_{(Q(t))^{-1}b(t)} \\
&= A(t)\underbrace{Q(t)C(t)}_{y(t)} + \underbrace{Q(t)(Q(t))^{-1}}_{I} b(t) \\
&= A(t) y(t) + b(t) \ \checkmark
\end{align*}

\begin{proof}
As \(Q\) is a fundamental matrix, \(\det(Q(t)) \neq 0\) for every \(t \in I\).

Then \(Q(t)z = b(t)\), for fixed \(t\), has a unique solution \(z = z(t)\).

Let \(C(t) = \int_{t_0}^t z(s) ds\) for \(t \in I\), \(t_0\) is some point in \(I\). Notice that \(z(t) = [Q(t)]^{-1} b(t)\).

Observe that, by construction, \(C\) is continuously differentiable and satisfies \(Q(t) \frac{dC}{dt} = b(t)\) for \(t \in I\). Let \(C = \begin{pmatrix} C_1 \\ \vdots \\ C_n \end{pmatrix}\). Only remains to prove that \(y(t) = C_1(t)y^{(1)}(t) + \cdots + C_n(t)y^{(n)}(t)\) solves \(\frac{dy}{dt} = A(t)y + b(t)\) in \(I\).
\end{proof}
