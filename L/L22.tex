\section{Lecture 22 - 12.25}

\begin{recap}[Review]
{\color{blue}\[a(t) \frac{d^2 y}{dt^2} + b(t) \frac{dy}{dt} + c(t) y = -\lambda r(t) y \ \alpha < t < \beta\]}
Boundaries conditions:
{\color{blue}\begin{align*}
B_\alpha[y] = a_1 y(\alpha) + b_1 \frac{dy}{dt}(\alpha) = 0 & \ |a_1| + |b_1| > 0 \\
B_\beta[y] = a_2 y(\beta) + b_2 \frac{dy}{dt}(\beta) = 0 & \ |a_2| + |b_2| > 0
\end{align*}}
\end{recap}

\begin{recap}[Recall]
Recall that we have considered the problem
{\color{blue}\[\begin{cases} \frac{d^2 y}{dt^2} + \lambda y = 0 & 0 < t < l \\ y(0) = 0 \\ y(l) = 0 \end{cases}\]}
Here \(L[y] = \lambda y\), \(-\frac{d^2 y}{dt^2}\). \(a=1, b=0, c=0, r=1\); \(a_1=1, b_1=0\); \(a_2=1, b_2=0\).

Eigenvalues:
{\color{blue}\[\lambda_n = \left(\frac{n\pi}{l}\right)^2 \ n=1, 2, \cdots\]}
\begin{itemize}
\item \(\lambda_n \in \mathbb{R} \ \forall n\)
\item \(\lambda_1 < \lambda_2 < \cdots\)
\item \(\lim_{n \to \infty} \lambda_n = +\infty\)
\end{itemize}

Eigenfunctions:
{\color{blue}\[y_n(t) = \sin\left(\frac{n\pi}{l}t\right), \ t \in [0, l]\]}
\begin{itemize}
\item \(\int_0^l y_n(t) y_m(t) \, dt = 0\) if \(n \neq m\).
\item \(f(t) = \sum_{n=1}^\infty B_n y_n(t)\), \(B_n = \frac{2}{l} \int_0^l f(t) y_n(t) \, dt\) for \(f \in C^2[0, l]\).
\end{itemize}

We've proven that this problem has nontrivial solutions given by multiples of \(y_n(t) = \sin(\frac{n\pi t}{l})\) (\(n=1, 2, \cdots\)), and that they correspond to \(\lambda_n = \left(\frac{n\pi}{l}\right)^2\).
\end{recap}

Also, we have introduced the notation
{\color{blue}\[L[y] = - \left( a(t) \frac{d^2 y}{dt^2} + b(t) \frac{dy}{dt} + c(t) y \right)\]}

Additionally, we say that any nontrivial solution to
{\color{blue}\[\begin{cases} L[y] = \lambda r y \\ B_\alpha[y] = B_\beta[y] = 0 \end{cases}\]}
is an eigenfunction with eigenvalue \(\lambda\).

\sout{Recall that if \(A\) is self-adjoint (or Hermitian) then \(A = A^*\), so \(Ax \cdot y = x \cdot Ay\) for all \(x, y \in \mathbb{R}^n\) if \(A \in \mathbb{R}^{n \times n}\).
We have introduced that \(L\) is self-adjoint in the space \(V = C^2[\alpha, \beta]\) if}
{\color{blue}\[(L[y], z) = (y, L[z]) \ \forall y, z \in V \tag{7}\]}
\sout{where \((u, w) = \int_\alpha^\beta u w \, dt\). From here, we are considering \(r=1\) for simplicity.
As an exercise, show that \(L[y] = -d^2y/dt^2\) satisfies (7).}

\begin{theorem}
{\color{blue}Assume that \(a \in C^2[\alpha, \beta]\) and \(b \in C^1[\alpha, \beta]\). The operator \(L\) is self-adjoint (in the sense of eq. (7)) if and only if \(b = \frac{da}{dt}\).}
\end{theorem}

\begin{remark}
\sout{Proof is commented out!}
\end{remark}

% \begin{proof}
% Let \(y, z \in V\). We have,
% \begin{align*}
% -(L[y], z) + (y, L[z]) &= \int_\alpha^\beta \left( a(t) \frac{d^2 y}{dt^2} + b(t) \frac{dy}{dt} + c(t) y \right) z \, dt - \int_\alpha^\beta \left( a(t) \frac{d^2 z}{dt^2} + b(t) \frac{dz}{dt} + c(t) z \right) y \, dt \\
% &= \int_\alpha^\beta \left( a(t) \frac{d^2 y}{dt^2} z + b(t) \frac{dy}{dt} z - a(t) \frac{d^2 z}{dt^2} y - b(t) \frac{dz}{dt} y \right) \, dt
% \end{align*}

% Integration by parts on the terms with second-order derivatives:
% \begin{align*}
% &= \left. \left( a(t) z \frac{dy}{dt} - a(t) y \frac{dz}{dt} \right) \right|_\alpha^\beta + \int_\alpha^\beta \left[ \frac{d(a(t)y)}{dt} \frac{dz}{dt} - \frac{d(a(t)z)}{dt} \frac{dy}{dt} + b(t) z \frac{dy}{dt} - b(t) y \frac{dz}{dt} \right] \, dt \\
% &= \left. \left( a(t) z \frac{dy}{dt} - a(t) y \frac{dz}{dt} \right) \right|_\alpha^\beta + \int_\alpha^\beta \left[ \left( \frac{da(t)}{dt} y \frac{dz}{dt} + a(t) \frac{dy}{dt} \frac{dz}{dt} \right) - \left( \frac{da(t)}{dt} z \frac{dy}{dt} + a(t) \frac{dz}{dt} \frac{dy}{dt} \right) \right. \\
% &\left. + b(t) z \frac{dy}{dt} - b(t) y \frac{dz}{dt} \right] \, dt \\
% &= \left. \left( a(t) z \frac{dy}{dt} - a(t) y \frac{dz}{dt} \right) \right|_\alpha^\beta + \int_\alpha^\beta \left( b(t) - \frac{da(t)}{dt} \right) \left( z \frac{dy}{dt} - y \frac{dz}{dt} \right) \, dt
% \end{align*}

% (We'll finish the proof tomorrow.)
% \end{proof}
