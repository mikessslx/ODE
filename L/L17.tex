\section{Lecture 17 - 12.09}

\begin{example}[Mechanical vibrations]

The goal is to find an eq. for \(y\).
By the Newton's second law,
\[m \frac{d^2 y}{dt^2} = F_{\text{total}}\]
Also, \(F_{\text{total}} = W + R + D + F\).

Here,
\begin{itemize}
\item \(W\) is the forced caused by the weight of the object, \(W > 0\), moreover, \(W = mg\), where \(g\) is the gravitational acceleration.

\item \(R\) is the restoring force. We know that \(R\) is proportional, by a constant \(k\), to the elongation or compression of the spring respect to its natural length, \(\Delta l + y\).
\begin{itemize}
\item If \(\Delta l + y > 0\), then \(R < 0\). Hence \(R = -k(\Delta l + y)\).
\item If \(\Delta l + y < 0\), then \(R > 0\). Hence \(R = -k(\Delta l + y)\).
\end{itemize}

Thus in either case, \(R = -k(\Delta l + y)\).

\item \(D\) is the damping force caused by the medium. Usually it acts in the direction opposite to the movement, and it is proportional to the magnitude of the velocity. (experimental law)
\begin{itemize}
\item If \(\frac{dy}{dt} > 0\), then \(D < 0\). Also, \(D = -c \frac{dy}{dt}\), where \(c > 0\) depending on the medium where the system is immersed.
\item Also, if \(\frac{dy}{dt} < 0\), then \(D > 0\) and \(D = -c \frac{dy}{dt}\).
\end{itemize}

\item \(F\) is an external forced in the vertical direction (it usually depends on \(t\)).
\end{itemize}

Thus,
\begin{align*}
m \frac{d^2 y}{dt^2} &= mg - k(\Delta l + y) - c \frac{dy}{dt} + F \\
&= \underbrace{mg - k \Delta l}_{=0} - ky - c \frac{dy}{dt} + F
\end{align*}
i.e.
\[m \frac{d^2 y}{dt^2} + c \frac{dy}{dt} + ky = F \tag{20}\]
\end{example}

\subsection{Free vibrations (\(c = 0, F \equiv 0\))}

Eq.(20) is \(m \frac{d^2 y}{dt^2} + ky = 0\), i.e. \(\frac{d^2 y}{dt^2} + \omega_0^2 y = 0\), \(\omega_0^2 = \frac{k}{m}\) (\(\omega_0 > 0\)).

The char. polynomial is \(p(\lambda) = \lambda^2 + \omega_0^2\). The roots are \(\omega_0 i\) and \(-\omega_0 i\). Then, the general solution is
\[y(t) = c_1 \cos(\omega_0 t) + c_2 \sin(\omega_0 t) \ t \geq 0, \ c_1, c_2 \in \mathbb{R}\]

Assume that \(c_1 \neq 0\). Then, \(y(t) = R \cos(\omega_0 t - \delta)\), where \(\delta \in (-\frac{\pi}{2}, \frac{\pi}{2})\), \(\tan \delta = \frac{c_2}{c_1}\) and \(R = \sqrt{c_1^2 + c_2^2}\) (exercise). Observe that \(y\) is bounded, more precisely, \(|y(t)| \leq R\) for all \(t \geq 0\). Also, \(y\) is periodic with period \(\frac{2\pi}{\omega_0}\). The solution behaves in the same way if \(c_1 = 0\). This type of motion is called simple harmonic motion. Also,
\begin{align*}
R &: \text{amplitude of the motion} \\
T_0 = \frac{2\pi}{\omega_0} &: \text{natural period of the motion} \\
\delta &: \text{phase angle of the motion} \\
\omega_0 = \sqrt{\frac{k}{m}} &: \text{natural frequency of the motion}
\end{align*}

\subsection{Damped free vibrations (\(c \neq 0, F \equiv 0\))}

Eq.(20) is \(m \frac{d^2 y}{dt^2} + c \frac{dy}{dt} + ky = 0\). The characteristic polynomial is
\[p(\lambda) = m \lambda^2 + c \lambda + k\]
The roots are given by \(\frac{-c \pm \sqrt{c^2 - 4mk}}{2m}\).

\begin{itemize}
\item Assume \(c^2 - 4mk > 0\). Then, the roots are
\[0 > \lambda_1 = \frac{-c + \sqrt{c^2 - 4mk}}{2m} \ \text{and} \ \lambda_2 = \frac{-c - \sqrt{c^2 - 4mk}}{2m} < 0\]
Hence, the general solution is
\[\underset{\to 0 \text{ as } t \to +\infty}{y(t)} = c_1 e^{\lambda_1 t} + c_2 e^{\lambda_2 t}, \ t \geq 0, \ c_1, c_2 \in \mathbb{R}\]

This motion is known as \textbf{overdamped motion}.

\item Assume that \(c^2 - 4mk = 0\).
The char. polynomial has a double root \(\lambda = -\frac{c}{2m}\).
The general solution is
\[\underset{\to 0 \text{ as } t \to +\infty}{y(t)} = c_1 e^{-\frac{c t}{2m}} + c_2 t e^{-\frac{c t}{2m}}, \ t \geq 0, \ c_1, c_2 \in \mathbb{R}\]
    
This motion is known as \textbf{critically damped motion}.

\item Assume that \(c^2 - 4mk < 0\).
The roots of the char. polynomial are
\[\lambda = -\frac{c}{2m} + i \frac{\sqrt{4mk - c^2}}{2m} \ \text{and} \ \bar{\lambda}\]
The general solution is
\[\underset{\to 0 \text{ as } t \to +\infty}{y(t)} = c_1 e^{-\frac{c t}{2m}} \cos(\mu t) + c_2 e^{-\frac{c t}{2m}} \sin(\mu t), \ t \geq 0, \ c_1, c_2 \in \mathbb{R},\]
where \(\mu = \frac{1}{2m} \sqrt{4mk - c^2}\).

This motion is called \textbf{damped vibrations}. Observe that, for \(c_1 \neq 0\),
\[y(t) = e^{-\frac{ct}{2m}} R \cos(\mu t - \delta) \ \text{and} \ y(t) = e^{-\frac{ct}{2m}} c_2 \cos(\mu t - \pi/2) \ \text{if } c_1 = 0.\]
\end{itemize}

\subsection{Damped forced vibrations (\(c \neq 0, F \neq 0\))}

The eq. is \(m \frac{d^2 y}{dt^2} + c \frac{dy}{dt} + ky = F\).
The general solution is given by
\[y(t) = y_H(t) + y_p(t)\]
where \(y_H\) is the general solution to the associated homogeneous eq. and \(y_p\) is a particular sol.

Consider, for instance, that \(F(t) = F_0 \cos(\omega t)\), where \(F_0\) and \(\omega\) are given.

It can be shown that
\[y_p(t) = \frac{F_0 \cos(\omega t - \delta)}{((k - m\omega^2)^2 + c^2 \omega^2)^{1/2}} \ \text{where } \tan \delta = \frac{c\omega}{k - m\omega^2}\]
is a particular solution provided \(\underbrace{k \neq m\omega^2}_{\omega \neq \sqrt{\frac{k}{m}} = \omega_0}\), in other words, \(\omega \neq \omega_0\).

As \(y_H(t) \to 0\) as \(t \to +\infty\), we observe that \(y(t)\) behaves as \(y_p(t)\) as \(t \to +\infty\).

\begin{xca}
Analyze the motion when \(\omega = \omega_0\).
\end{xca}

\subsection{Forced free vibrations (\(c = 0, F \not\equiv 0\))}

The eq. is \(m \frac{d^2 y}{dt^2} + ky = F\). The general sol is \(y(t) = y_H(t) + y_p(t)\), \(y_H, y_p\) as before.

Consider again the case where \(F(t) = \underbrace{F_0 \cos(\omega t)}_{\text{bounded}}\), \(F_0, \omega\) given.

If \(\omega \neq \omega_0\), then it can be shown that
\[y_p(t) = \frac{F_0 \cos(\omega t)}{m(\omega_0^2 - \omega^2)}\]
is a particular solution. So,
\[y(t) = R \cos(\omega_0 t - \delta) + \frac{F_0 \cos(\omega t)}{m(\omega_0^2 - \omega^2)}, \ t \geq 0, \ c_1, c_2 \in \mathbb{R}\]

Observe that, as before, \(|y(t)| \leq M\) for every \(t \geq 0\), as in the previous case. And, \(y\) follows an oscillatory law.

Finally, assume that \(\omega = \omega_0\). It can be shown that
\[y_p(t) = \frac{F_0 t}{2m\omega_0} \sin(\omega_0 t)\]
is a particular solution.
Then,
\[y(t) = \underbrace{R \cos(\omega_0 t - \delta)}_{\text{bounded as } t \to +\infty} + \underbrace{\frac{F_0 t}{2m\omega_0} \sin(\omega_0 t)}_{\to \pm \infty \text{ as } t \to +\infty} \ t \geq 0\]

Here we say that the external force is in resonance with the natural frequency of the system.
