\section{Lecture 4 - 10.23}

\subsection{\S2. The Existence and Uniqueness Theorem for First-Order ODEs (Continued)}

Consider the initial value problem (IVP):
{\color{blue}\[\begin{cases}
\frac{dy}{dt} = f(t, y) \\
y(t_0) = y_0
\end{cases}\]}

\begin{example}
{\color{blue}\[\begin{cases}
\frac{dy}{dt} = 1 + y^2 \\
y(0) = 0
\end{cases}\]}

Here, \(f(t, y) = 1 + y^2\). This IVP admits a solution given by \(y(t) = \tan(t)\) for \(t \in \left( -\frac{\pi}{2}, \frac{\pi}{2} \right)\). {\color{blue}Notice that \(y\) cannot be extended outside this interval as a continuous function.}
\end{example}

\begin{remark}
The general solution to \(\frac{dy}{dt} = 1 + y^2\) is \(y(t) = \tan(t + c)\). This function satisfies \(y(0) = 0\) if and only if \(\tan(c) = 0\), so \(c = k\pi\) for some \(k \in \mathbb{Z}\). If \(c = k\pi\) is fixed, the domain of \(y\) can be selected as \(\left( -\frac{\pi}{2}, \frac{\pi}{2} \right)\).
\end{remark}

\begin{example}
{\color{blue}\[\begin{cases}
\frac{dy}{dt} = \sqrt{|y|} \\
y(0) = 0
\end{cases}\]}

Here, \(f(t, y) = \sqrt{|y|}\). {\color{blue}This problem has infinitely many solutions, for instance, those of the form
\[y(t) = 
\begin{cases} 
0 & t \geq c \\
\frac{(t - c)^2}{4} & t < c 
\end{cases}\]
for any \(c \geq 0\).}
\end{example}

\begin{theorem}[Picard-Lindelöf Theorem]
{\color{blue}Assume \(f\) and \(\frac{\partial f}{\partial y}\) are continuous in the rectangle
\[R = \left\{ (t, y) \mid t_0 \leq t \leq t_0 + a, \, |y - y_0| \leq b \right\},\]
where \(a, b\) are given numbers.

Let \(M = \max_{(t, y) \in R} |f(t, y)|\) and \(\alpha = \min\left\{ a, \frac{b}{M} \right\}\). Then, the initial value problem
\[\begin{cases}
\frac{dy}{dt} = f(t, y) \\
y(t_0) = y_0
\end{cases}\]
has a unique solution in \([t_0, t_0 + \alpha]\). A similar result holds for intervals of the form \([t_0 - \alpha, t_0]\).}
\end{theorem}

\begin{remark}
\sout{Proof is commented out!}
\end{remark}

% \begin{proof}
% The general idea is to construct a sequence of functions \(\{ y_n \}\) such that, as \(n \to \infty\), \(y_n\) is ``near'' to solving the problem. Then, prove there exists a limit function to which \(\{ y_n \}\) converges, show the limit function solves the problem, and finally prove that it is the unique solution.

% The proof is split into several steps.

% \textbf{Step 1: Definition of the Sequence \(\{y_n\}\)}

% A continuous function \(y\) is a solution to the initial value problem (1)-(2) if and only if y is a solution to
% \[y(t) = y_0 + \int_{t_0}^t f(s, y(s)) \, ds \tag{6}\]

% \begin{itemize}
% \item If \(y\) solves (1)-(2), it is continuous. Integrating \(\frac{dy}{ds} = f(s, y(s))\) and using \(y(t_0) = y_0\), we get
% \begin{align*}
% \int_{t_0}^t \frac{dy(s)}{ds} \, ds = \int_{t_0}^t f(s, y(s)) \, ds 
% &\Longleftrightarrow y(t) - y_0 = \int_{t_0}^t f(s, y(s)) \, ds \\
% &\Longleftrightarrow y(t) = y_0 + \int_{t_0}^t f(s, y(s)) \, ds
% \end{align*}

% \item Conversely, if \(y\) is a continuous solution to (6), then \(y(t_0) = y_0\), and also differentiating (6) gives 
% \[\frac{dy}{dt} = f(t, y(t))\]
% \end{itemize}

% We'll solve (1)-(2) by solving (6) for a continuous function by defining the sequence \(\{y_n\}\) as
% \[\begin{cases}
% y_0(t) = y_0 \\
% y_n(t) = y_0 + \int_{t_0}^t f(s, y_{n-1}(s)) \, ds \ (n = 1, 2, \cdots)
% \end{cases}\]
% \end{proof}
