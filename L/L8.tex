\section{Lecture 8 - 11.06}

\subsection{§3. Systems of First-Order Equations, \(n\)-th Order Equations}

\begin{definition}[System of First-Order ODEs]
    A system of first-order ODEs is written as
    \[\begin{cases}
        \frac{dy_1}{dt} = f_1(t, y_1, y_2, \cdots, y_n) \\
        \frac{dy_2}{dt} = f_2(t, y_1, y_2, \cdots, y_n) \\
        \vdots \\
        \frac{dy_n}{dt} = f_n(t, y_1, y_2, \cdots, y_n)
    \end{cases} \tag{1}\]
    where \(f_1, f_2, \cdots, f_n\) are given functions. We'll usually consider systems with an initial condition in the form
    \[y_1(t_0) = y_1^0, \ y_2(t_0) = y_2^0, \ \cdots, \ y_n(t_0) = y_n^0 \tag{2}\]
    where \(y_1^0, \cdots, y_n^0 \in \mathbb{R}\) are known.

    We say that \(y_1, \cdots, y_n\) is a solution to the initial-value problem (1)-(2) on \(I\) if:
    \[\begin{cases}
        \frac{dy_1}{dt}(t) = f_1(t, y_1(t), y_2(t), \cdots, y_n(t)) \\
        \frac{dy_2}{dt}(t) = f_2(t, y_1(t), \cdots, y_n(t)) \\
        \vdots \\
        \frac{dy_n}{dt}(t) = f_n(t, y_1(t), \cdots, y_n(t))
    \end{cases} \ \forall \ t \in I\]
    and \(y_1, \cdots, y_n\) satisfy (2).
\end{definition}

\begin{example}
    \leavevmode
    \begin{enumerate}[a)]
        \item
        The system
        \[\begin{cases}
            \frac{dy_1}{dt} = a y_1 - \alpha y_1 y_2 \\
            \frac{dy_2}{dt} = -c y_2 + \gamma y_1 y_2
        \end{cases}\]
        is the Lotka-Volterra system (or predator-prey system). Here \(y_1, y_2\) are the populations of prey and predators, respectively; \(a, \alpha, c, \gamma > 0\).

        \item
        \[\begin{cases}
            \frac{dg}{dt} = -m_1 g - m_2 h + J(t) \\
            \frac{dh}{dt} = -m_3 h + m_4 g
        \end{cases}\]
        models the variations of the blood glucose \(g\) and the net hormonal concentration \(h\) (respect to their optimal values). Here \(m_1, m_2, m_3, m_4 > 0\) and \(J(t)\) are known.
    \end{enumerate}
\end{example}

\begin{remark}
    Let
    \[\vec{y} = \begin{pmatrix} y_1 \\ y_2 \\ \vdots \\ y_n \end{pmatrix}, \ \frac{d\vec{y}}{dt} = \begin{pmatrix} dy_1/dt \\ dy_2/dt \\ \vdots \\ dy_n/dt \end{pmatrix}, \ \vec{F} = \begin{pmatrix} f_1 \\ f_2 \\ \vdots \\ f_n \end{pmatrix}, \ \vec{y}^0 = \begin{pmatrix} y_1^0 \\ y_2^0 \\ \vdots \\ y_n^0 \end{pmatrix}\]

    Then the system (1)-(2) is
    \[\begin{cases}
        \frac{d\vec{y}}{dt} = \vec{F}(t, \vec{y}) \\
        \vec{y}(t_0) = \vec{y}^0
    \end{cases}\]

    The theorem of existence and uniqueness of solution for the initial-value problem for systems is obtained exactly as for equations. Now, we replace the assumption on \(\frac{\partial f}{\partial y}\) (for the case of equations) by the assumption that \(D_{\vec{y}} \vec{F}\) is continuous, where
    \[D_{\vec{y}} \vec{F} = \begin{pmatrix} \nabla_y f_1 \\ \nabla_y f_2 \\ \vdots \\ \nabla_y f_n \end{pmatrix} = \begin{pmatrix} \frac{\partial f_1}{\partial y_1} & \frac{\partial f_1}{\partial y_2} & \cdots & \frac{\partial f_1}{\partial y_n} \\
    \frac{\partial f_2}{\partial y_1} & \frac{\partial f_2}{\partial y_2} & \cdots & \frac{\partial f_2}{\partial y_n} \\
    \vdots & \vdots & \ddots & \vdots \\
    \frac{\partial f_n}{\partial y_1} & \frac{\partial f_n}{\partial y_2} & \cdots & \frac{\partial f_n}{\partial y_n}
    \end{pmatrix}\]

    The existence theorem also holds in the context of systems.
\end{remark}

\begin{remark}
    The theory of continuation of solutions also applies to systems with minor changes in the proof. The general form of a linear system is
    \[\begin{cases}
        \frac{dy_1}{dt} = a_{11}(t)y_1 + a_{12}(t)y_2 + \cdots + a_{1n}(t)y_n + b_1(t) \\
        \frac{dy_2}{dt} = a_{21}(t)y_1 + a_{22}(t)y_2 + \cdots + a_{2n}(t)y_n + b_2(t) \\
        \vdots \\
        \frac{dy_n}{dt} = a_{n1}(t)y_1 + a_{n2}(t)y_2 + \cdots + a_{nn}(t)y_n + b_n(t)
    \end{cases}\]

    Let
    \[A(t) = \begin{pmatrix} a_{11}(t) & \cdots & a_{1n}(t) \\
    a_{21}(t) & \cdots & a_{2n}(t) \\
    \vdots & \ddots & \vdots \\
    a_{n1}(t) & \cdots & a_{nn}(t)
    \end{pmatrix}, \ \vec{b}(t) = \begin{pmatrix} b_1(t) \\ b_2(t) \\ \vdots \\ b_n(t) \end{pmatrix}\]

    Then the linear system is
    \[\frac{d\vec{y}}{dt} = A(t)\vec{y} + \vec{b}(t)\]

    If \(\vec{b} \equiv 0\), we say that the system is homogeneous; otherwise, we say that the system is non-homogeneous.

    We'll focus on Linear Systems.
    \begin{itemize}
        \item For \(\frac{d\vec{y}}{dt} = A(t)\vec{y}\) (homogeneous):
        \begin{itemize}
            \item The set of solutions is a vector space of dimension \(n\).
            \item The general solution is:
            \[\vec{y}_h(t) = \alpha_1 \vec{y}^{(1)}(t) + \cdots + \alpha_n \vec{y}^{(n)}(t)\]
            where \(\{\vec{y}^{(1)}(t), \cdots, \vec{y}^{(n)}(t)\}\) is a basis (fundamental set of solutions).
        \end{itemize}
        \item For \(\frac{d\vec{y}}{dt} = A(t)\vec{y} + \vec{b}(t)\) (\(\vec{b} \neq 0\), non-homogeneous):
        \begin{itemize}
            \item The general solution is:
            \[\vec{y}_h(t) + \vec{y}_p(t)\]
            where \(\vec{y}_p\) is one particular solution to the non-homogeneous system.
        \end{itemize}
        \item Qualitative Properties of Nonlinear Systems (some of them) can be deduced from associated linear systems.
    \end{itemize}
\end{remark}
