\section{Lecture 21 - 12.23}

\subsection{\S6. Sturm-Liouville Theory for second-order boundary problems}

\begin{example}
Let's look for a solution to
\[\begin{cases}
u_t - k u_{xx} = 0 & \text{(PDE)} \ 0 < x < L, \ t > 0 \ (k > 0) \\
u(0, t) = u(L, t) = 0 & \text{(BC)} \ t \geq 0 \\
u(x, 0) = f(x) & \text{(IC)} \ 0 \leq x \leq L
\end{cases}\]
where \(u(x, t)\): temperature at position \(x\) at time \(t > 0\).
\end{example}

\begin{proof}[Solution]
First, look for a solution \(u\) of the PDE and the BC.
Specifically, let's look for a solution in the form \(u(x, t) = X(x) T(t)\).
Replacing \(u\) in the PDE, we get (formally)
\begin{align*}
X T' - k X'' T &= 0 \\
\underbrace{\frac{1}{k} \frac{T'}{T}}_{\text{function of } t} = \underbrace{\frac{X''}{X}}_{\text{function of } x} &= -\lambda \ \text{for some } \lambda \in \mathbb{R}
\end{align*}

Then \(X'' + \lambda X = 0 \ 0 < x < L\) and \(T' + k \lambda T = 0 \ t > 0\).
Also,
\[u(0, t) = 0 \Rightarrow X(0) T(t) = 0 \ \forall t > 0\]

Moreover, we look for a nontrivial solution. So, \(X\) and \(T\) are not identically zero. Then \(X(0) = 0\). Similarly, \(X(L) = 0\).

Therefore, second-order ODE with BC:
\[\begin{cases}
X'' + \lambda X = 0 & 0 < x < L \\
X(0) = X(L) = 0
\end{cases} \ \text{and} \ T' + \lambda k T = 0 \ t > 0.\]

Let's find, if possible, nontrivial solutions to these problems. Let's start with the problem for \(X\).

\begin{itemize}
\item Assume \(\lambda = 0\)

The ODE is \(X'' = 0 \ 0 < x < L\).
The general sol is \(X(x) = Ax + B, \ A, B \in \mathbb{R}\).

Also, \(X(0) = B = 0\), \(X(L) = AL = 0\) so \(A = 0\).
Then, \(X \equiv 0\).

\item Assume \(\lambda < 0\)

The char. polynomial is \(p(\omega) = \omega^2 + \lambda\).
The roots are \(\omega = \sqrt{-\lambda}\) and \(\omega = -\sqrt{-\lambda}\).

The general sol to the ODE is \(X(x) = A e^{\sqrt{-\lambda}x} + B e^{-\sqrt{-\lambda}x}\).

Also, \(X(0) = A + B = 0 \Rightarrow A = -B\). \(X(L) = A(e^{\sqrt{-\lambda}L} - e^{-\sqrt{-\lambda}L}) = 0 \Rightarrow A = 0\).
Then \(X \equiv 0\).

\item Assume \(\lambda > 0\)

The roots of the char. polynomial are \(\omega = \sqrt{\lambda}i\) and \(\omega = -\sqrt{\lambda}i\).

The general solution is \(X(x) = A \cos(\sqrt{\lambda}x) + B \sin(\sqrt{\lambda}x)\).

Also, \(X(0) = A = 0\). \(X(L) = B \sin(\sqrt{\lambda}L) = 0\). To have a nontrivial sol, we consider \(B \neq 0\).
Then \(\sin(\sqrt{\lambda}L) = 0\), i.e. \(\sqrt{\lambda}L = n\pi\), i.e. \(\lambda_n = (\frac{n\pi}{L})^2\), \(n=1, 2, \cdots\).

So \(X_n(x) = \sin(\frac{n\pi}{L}x)\) is a nontrivial solution, \(n=1, 2, \cdots\).
\end{itemize}

For \(\lambda_n = (\frac{n\pi}{L})^2\), let's solve the ODE for \(T\): \(T' + k \lambda_n T = 0\). The function
\[T_n(t) = e^{-k \lambda_n t}\]
is a nontrivial solution (exercise).

Then, \(u_n(x, t) = e^{-k (\frac{n\pi}{L})^2 t} \sin(\frac{n\pi}{L}x) \ n=1, 2, \cdots\)
are sol. to the PDE and the BC.

To have a solution to the complete problem, including the IC, we look for a solution in the form
\[u(x, t) = \sum_{n=1}^\infty B_n u_n(x, t)\]
Formally, \(u\) solves the PDE and the BC. Also,
\[u(x, 0) = \sum_{n=1}^\infty B_n \sin(\frac{n\pi}{L}x) = f(x) \tag{*}\]
Assume that \(f\) admits a representation in the form (*). Let's look for \(B_n\).
\[\int_0^L f(x) \sin(\frac{\pi n}{L}x) dx = \sum_{m=1}^\infty B_m \int_0^L \sin(\frac{\pi n}{L}x) \sin(\frac{\pi m}{L}x) dx \overset{\text{delicate step}}{=} \cdots\]
\end{proof}

\begin{example}
Consider the problem
\begin{align*}
a(t) \frac{d^2 y}{dt^2} + b(t) \frac{dy}{dt} + c(t) y &= -\lambda r(t) y \ \alpha < t < \beta \\
B_\alpha[y] &= a_1 y(\alpha) + b_1 \frac{dy}{dt}(\alpha) = 0 \\
B_\beta[y] &= a_2 y(\beta) + b_2 \frac{dy}{dt}(\beta) = 0
\end{align*}

Let \(V\) be the space of twice continuously diff functions in \([\alpha, \beta]\). Also,
\[L[y] = -\left( a(t) \frac{d^2 y}{dt^2} + b(t) \frac{dy}{dt} + c(t) y \right)\]
\end{example}

\begin{remark}
We know functions \(f: \mathbb{R} \to \mathbb{R}\), \(f(x) = y\).
We usually call operators to functions whose domain is a set of functions.
If \(L: V \to W\) where \(W\) is the space of cont. functions in \([\alpha, \beta]\),
\(L[y] = -\frac{d^2 y}{dt^2}\), then we call \(L\) an operator.
\end{remark}

If \(r\) is a nonnegative function then any nontrivial solution to
\[\begin{cases} L[y] = \lambda r y & \alpha < t < \beta \\ B_\alpha[y] = B_\beta[y] = 0 \end{cases}\]
is called an \textbf{eigenfunction} of \(L\) with eigenvalue \(\lambda\).
Also, the function \(r\) is called a \textbf{weight}.
We'll work with the inner product
\[(y, z) = \int_\alpha^\beta r y z \, dt \ y, z \in V\]

\begin{example}
If \(r=1, a=1, b=0, c=0\), \(b_1=b_2=0, a_1=a_2, \alpha=0, \beta=L\),
\begin{align*}
-\frac{d^2 y}{dt^2} &= \lambda y \ 0 < t < L \\
y(0) &= y(L) = 0
\end{align*}
then we know that the eigenvalues are \(\lambda_n = (\frac{n\pi}{L})^2\) and the eigenfunctions are \(y_n(t) = \sin(\frac{n\pi}{L}t)\), \(n=1, 2, \cdots\).
\end{example}

\begin{recap}[A brief review of properties of self-adjoint matrices]
Let \(A = (a_{ij})\), \(i,j=1, \cdots, n\), \(a_{ij} \in \mathbb{C}\), \(A \in \mathbb{C}^{n \times n}\).
\begin{itemize}
\item The adjoint of \(A\) is \(A^* = \bar{A}^t = (\overline{a_{ji}})\).
\item We say that \(A\) is \textbf{self-adjoint} or Hermitian if \(A^* = A\).
\item If \(A\) is self-adjoint then:
\begin{enumerate}[i)]
\item All eigenvalues of \(A\) are real.
\item There exists a basis of \(\mathbb{R}^n\) of eigenvectors of \(A\).
\item If \(v\) and \(w\) are eigenvector with different eigenvalues then \(v \cdot w = 0\).
\item \((Av, w) = (v, A^* w) \ \forall v, w \in \mathbb{C}^n\).
\end{enumerate}
\end{itemize}
\end{recap}
