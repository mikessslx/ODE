\section{Lecture 6 - 10.30}

\subsection{Review: Picard-Lindelöf Theorem}

\begin{recap}
\textbf{Picard Iterates (Sequence of Successive Approximations)}
    
The sequence \(\{y_n(t)\}\) is defined as:
\[\begin{cases}
y_0(t) = y_0 \\
y_{n+1}(t) = y_0 + \int_{t_0}^t f(s, y_n(s)) \, ds \ (n = 0, 1, 2, \cdots)
\end{cases}\]

This sequence \(\{y_n(t)\}\) converges to a continuous function \(y(t)\) for each \(t \in [t_0, t_0 + \alpha]\), and \(y(t)\) solves the initial value problem on \([t_0, t_0 + \alpha]\).
\end{recap}

\subsubsection{Proof (Continued)}

\begin{remark}
Proof is commented out!
\end{remark}

% \begin{proof}[Proof (Continued)]
% \textbf{Step 4}: We'll prove that the limit function \(y\) is unique.

% Assume \(y\) and \(z\) are two solutions to the IVP on \([t_0, t_0 + \alpha]\). Then,
% \[y(t) = y_0 + \int_{t_0}^t f(s, y(s)) \, ds, \ z(t) = y_0 + \int_{t_0}^t f(s, z(s)) \, ds\]

% Taking the absolute difference,
% \[|y(t) - z(t)| = \left| \int_{t_0}^t \left( f(s, y(s)) - f(s, z(s)) \right) ds \right| \leq \int_{t_0}^t \left| f(s, y(s)) - f(s, z(s)) \right| ds\]

% By the Mean Value Theorem, \(|f(s, y(s)) - f(s, z(s))| \leq L |y(s) - z(s)|\) (where \(L = \max_{(t,y) \in R} \left| \frac{\partial f}{\partial y}(t, y) \right|\)).

% We have
% \[|y(t) - z(t)| \leq L \int_{t_0}^t |y(s) - z(s)| ds\]

% We'll now prove a general result that will give us that \(z = y\).

% Assume that \(w\) is a non-negative and continuous function in \([t_0, t_0 + \alpha]\) such that
% \[w(t) \leq L \int_{t_0}^t w(s) ds\]

% Then, \(w \equiv 0\). Let \(W(t) = \int_{t_0}^t w(s) ds\), \(t \in [t_0, t_0 + \alpha]\).

% We have that (in \([t_0, t_0 + \alpha]\))
% \[\frac{dW}{dt} = w(t) \leq L \int_{t_0}^t w(s) ds = L W(t) \implies \frac{dW}{dt} - L W(t) \leq 0\]

% Multiplying by \(e^{-L(t - t_0)}\) (a positive function),
% \[\frac{d}{dt} \left( e^{-L(t - t_0)} W(t) \right) \leq 0\]

% Since \(e^{-L(t - t_0)} W\) is non-negative, zero at \(t = t_0\) and decreasing, it follows that \(e^{-L(t - t_0)} W(t) = 0\) for every \(t \in [t_0, t_0 + \alpha]\). As \(e^{-L(t - t_0)} > 0\) for all \(t\), we get that \(W \equiv 0\) in \([t_0, t_0 + \alpha]\). Only, remains to note that this implies that \(w \equiv 0\). In particular, for \(w = |y - z|\), we have \(y = z\), so the solution is unique.
% \end{proof}

\begin{remark}
\leavevmode
\begin{enumerate}[a)]
\item
In the proof, we repeatedly used the inequality:
\[|f(s, y) - f(s, z)| \leq L |y - z| \ \forall (s, y), (s, z) \in R \tag{*}\]

The theorem’s conclusion holds if we replace the assumption that \(\frac{\partial f}{\partial y}\) is continuous in \(R\) with \(f\) satisfying inequality (*). A function satisfying (*) is called \textbf{Lipschitz} with respect to its second argument. There exist differentiable Lipschitz functions (e.g. \(f(t, y) = 2y\), which is differentiable and satisfies \(|2y - 2z| = 2|y - z|\), so \(L = 2\)).

\item
If we remove the assumption on \(\frac{\partial f}{\partial y}\), the \textbf{Cauchy-Peano Theorem} guarantees that the initial value problem has at least one solution.
\end{enumerate}
\end{remark}
