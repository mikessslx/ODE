\section{Lecture 13 - 11.25}

\begin{example}
Find the general solution to
\[\frac{dy}{dt} = y - 2.\]

\paragraph*{Method 1} (Method of integrating factors)
    
\sout{\(\frac{dy}{dt} - y = -2\) is in the form \(\frac{dy}{dt} + ay = b\) for \(a = -1\) and \(b = -2\).}

\sout{The eq. is linear with cont. coeff in \(\mathbb{R}\) (const. coeff.). Then the general solution is}
\begin{align*}
y(t) &= e^{-\int a dt} \left( \int b e^{\int a dt} dt + C \right), \ t \in \mathbb{R}, \ C \in \mathbb{R} \\
&= e^t \left( -2 \int e^{-t} dt + C \right) = e^t (2e^{-t} + C) = 2 + Ce^t
\end{align*}
    
Now, \(y'(t) = Ce^t = y(t) - 2\), \(t \in \mathbb{R}\).
{\color{blue}Then the solution is \(y(t) = Ce^t + 2\) for \(t \in \mathbb{R}\).}
\end{example}

\begin{example}
Find the solution to
\[\frac{dy}{dt} = y - 2, \ y(0) = 10.\]
    
The solution exists by the Picard-Lindel\"of thm (check the assumptions!). Then, there exists a neighborhood of \(0\), \(E\), s.t. \(y(t) > 2\) for \(t \in E\).
    
\paragraph*{Method 2} (Separable equation)
\begin{align*}
\frac{dy}{dt} = y - 2 \iff y'(t) = y(t) - 2 &\overset{?}{\rightsquigarrow} \int \frac{y'(t) dt}{y(t) - 2} = \int 1 dt \\
&\implies \int \frac{1}{s-2} ds = t + C, \ y(t) = s, \ y'(t) dt = ds \\
&\implies \ln|s-2| = t + C \implies |y(t) - 2| = e^C e^t \\
&\overset{?}{\rightsquigarrow} y(t) - 2 = \tilde{C} e^t \implies y(t) = \tilde{C} e^t + 2 \ \text{formal solution.}
\end{align*}
\end{example}

\begin{recap}
\begin{itemize}
\item \sout{\(\frac{dy}{dt} = Ay\), where \(A \in \mathbb{R}^{n \times n}\).}
\item \sout{If \(A\) admits \(n\) real distinct eigenvalues \(\lambda_1, \cdots, \lambda_n\) with associated eigenvectors \(v_1, \cdots, v_n \in \mathbb{R}^n\), then \(\{ v_1 e^{\lambda_1 t}, \cdots, v_n e^{\lambda_n t} \}\) is a basis of solutions, \(t \in \mathbb{R}\).}
\item \sout{If \(A\) admits an eigenvalue \(\lambda = \alpha + i\beta\) where \(\beta \neq 0\), with eigenvector \(v = u + iw, \ u, w \in \mathbb{R}^n\), then}
\[e^{\alpha t} (u \cos(\beta t) - w \sin(\beta t)) \ \text{and} \ e^{\alpha t} (u \sin(\beta t) + w \cos(\beta t)), \ t \in \mathbb{R}\]
\sout{are two l.i. solutions.}
\end{itemize}
\end{recap}

\subsection{Case 3: Eigenvalues with multiplicity greater than 1}

For instance, the matrix \(A = \begin{pmatrix} 1 & 1 & 0 \\ 0 & 1 & 0 \\ 0 & 0 & 2 \end{pmatrix}\) admits two real eigenvalues: \(\lambda_1 = 1\) with multiplicity 2 and \(\lambda_2 = 2\) with multiplicity 1. Also, the eigenvectors for \(\lambda_1\) form a space of dimension 1, and the eigenvector for \(\lambda_2\) as well.

Let's introduce the matrix exponential. We define
{\color{blue}\[e^A = I + A + \frac{A^2}{2} + \frac{A^3}{3!} + \frac{A^4}{4!} + \cdots \ \text{where } A \in \mathbb{R}^{n \times n}\]}

\begin{remark}
This is a well-definition since the r.h.s. is convergent in the space of matrices in \(\mathbb{R}^{n \times n}\) with the norm defined by \(\|A\| = \left( \sum_{i,j=1}^n a_{ij}^2 \right)^{1/2}\) where \(A = (a_{ij})_{i,j=1, \cdots, n}\).
\end{remark}

Consider the map \(t \mapsto e^{At} = I + At + \frac{A^2 t^2}{2} + \frac{A^3 t^3}{3!} + \cdots\).
It can be shown that this map can be differentiated term-by-term. Then,
\begin{align*}
\frac{d(e^{At})}{dt} &= \frac{d}{dt} \left( I + At + \frac{A^2 t^2}{2} + \frac{A^3 t^3}{3!} + \cdots \right) \\
&= A + A^2 t + \frac{A^3 t^2}{2!} + \cdots \\
&= A \left( I + At + \frac{A^2 t^2}{2!} + \cdots \right) = A e^{At}
\end{align*}
{\color{blue}\[\frac{d(e^{At})}{dt} = A e^{At}\]}

\begin{remark}
\sout{\(F: I \subset \mathbb{R} \to \mathbb{R}^n, \ F = (f_1, \cdots, f_n), \ f_k: I \subset \mathbb{R} \to \mathbb{R}\) and \(\frac{dF}{dt} = \left( \frac{df_1}{dt}, \cdots, \frac{df_n}{dt} \right)\).}
\end{remark}

{\color{blue}Then, if \(v \in \mathbb{R}^n\) then \(y(t) = e^{At} v\) solves \(\frac{dy}{dt} = Ay\).}

\begin{xca}
Can we find vectors \(v \in \mathbb{R}^n\) s.t. \(e^{At}v\) is finite?

In other words, s.t. \(Iv + Atv + \frac{A^2 t^2}{2!}v + \cdots\) can be explicitly determined?
\end{xca}
\begin{proof}[Solution]
Assume that \(v \in \mathbb{R}^n\) s.t. \((A - \lambda I)^m v = 0\) for some \(m \in \mathbb{N}\). Then, \((A - \lambda I)^{m+l} v = 0\) for any \(l \in \mathbb{N}\). Then,
\[e^{(A-\lambda I)t} v = \left( I + (A-\lambda I)t v + \frac{(A-\lambda I)^2 t^2}{2} v + \cdots + \frac{(A-\lambda I)^{m-1} t^{m-1}}{(m-1)!} v \right)\]

We have
\begin{align*}
e^{\lambda I t} v &= Iv + (\lambda I)v t + \frac{(\lambda I)^2 v t^2}{2} + \cdots \\
&= v + \lambda v t + \frac{\lambda^2 v t^2}{2} + \cdots \\
&= \left( 1 + \lambda t + \frac{\lambda^2 t^2}{2} + \cdots \right) v = e^{\lambda t} v
\end{align*}

Then,
\begin{align*}
y(t) = e^{At} v &= e^{(At - \lambda I t) + (\lambda I t)} v \\
&= e^{(A - \lambda I)t} e^{\lambda I t} v \\
&= e^{\lambda t} \left( e^{(A-\lambda I)t} v \right) \\
&= e^{\lambda t} \left( I + (A-\lambda I)t + \frac{(A-\lambda I)^2 t^2}{2} + \cdots + \frac{(A-\lambda I)^{m-1} t^{m-1}}{(m-1)!} \right) v \tag{\(*\)}
\end{align*}
{\color{blue}\[y(t) = e^{At} v = e^{\lambda t} \left( I + (A-\lambda I)t + \frac{(A-\lambda I)^2 t^2}{2} + \cdots + \frac{(A-\lambda I)^{m-1} t^{m-1}}{(m-1)!} \right) v \tag{\(*\)}\]}

\begin{note}
Recall that \(e^{A+B} = e^A e^B\) this is valid iff \(AB = BA\).
\end{note}

If there exist \(m \in \mathbb{N}\) and \(v \in \mathbb{C}^n\) s.t. \((A-\lambda I)^m v = 0\) then we can compute \(y(t) = e^{At}v, \ t \in \mathbb{R}\), by the formula (\(*\)).
\end{proof}

If \(\lambda\) is an eigenvalue of \(A\) with multiplicity \(l\), then there exist \(m \le l\) such that the equation \((A-\lambda I)^m v = 0\) admits at least \(l\) linearly independent solutions in \(\mathbb{C}^n\).

Therefore, if \(\lambda \in \mathbb{C}\) is an eigenvalue of \(A\) with multiplicity \(l\), then there exist \(v_1, \cdots, v_l \in \mathbb{C}^n\) which are l.i. and hence
{\color{blue}\[
y^{(1)}(t) = e^{At} v_1, \ \cdots, \ y^{(l)}(t) = e^{At} v_l, \ t \in \mathbb{R}
\]
are l.i. solutions to the system \(\frac{dy}{dt} = Ay\).}

\begin{example}
Assume that \(A \in \mathbb{R}^{5 \times 5}\) with eigenvalues \(\lambda_1 = 2, \ \lambda_2 = 1+i, \ \lambda_3 = \overline{\lambda_2}\), and \(\lambda_4 = 3\) with multiplicity 2.
    
\begin{itemize}
\item {\color{blue}From \(\lambda_1 = 2\), we get the solution \(y^{(1)}(t) = v_1 e^{2t}\) where \(v_1\) is an eigenvector for \(\lambda_1\).}
\item {\color{blue}From \(\lambda_2 = 1+i\) we get two solutions \(y^{(2)}(t) = e^{t}(u \cos t - w \sin t)\) and \(y^{(3)}(t) = e^{t}(u \sin t + w \cos t)\) in \(\mathbb{R}\), where \(v = u + iw\) is an eigenvector for \(\lambda_2\), \(u, w \in \mathbb{R}^5\).}
\item {\color{blue}From \(\lambda_4 = 3\) we get the solutions \(y^{(4)}(t) = v_4 e^{3t}\), where \(v_4\) is an eigenvector for \(\lambda_4\), and \(y^{(5)}(t) = e^{At} \tilde{v}\), \(t \in \mathbb{R}\), where \(\tilde{v}\) is a solution to \((A-3I)^m \tilde{v} = 0\) for some \(m \in \mathbb{N}\), \(m \le 2\).}
\end{itemize}
    
Let \(\mathcal{B} = \{ y^{(1)}, y^{(2)}, y^{(3)}, y^{(4)}, y^{(5)} \}\). This is a basis if \(\mathcal{B}\) is l.i.
\end{example}
