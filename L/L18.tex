\section{Lecture 18 - 12.11}

\subsection{Linear equations with non constant coefficients}

Consider
\[p(t) \frac{d^2 y}{dt^2} + q(t) \frac{dy}{dt} + r(t) y = f(t)\]
where \(p, q, r, \text{and} f\) are continuous functions in some open interval \(I\).

The analysis is affected by the zeros of the function \(p(t)\).

\subsubsection{Case 1: \(p(t) \neq 0\) for every \(t \in I\)}

The eq. is equivalent to
\[\frac{d^2 y}{dt^2} + \frac{q(t)}{p(t)} \frac{dy}{dt} + \frac{r(t)}{p(t)} y = \frac{f(t)}{p(t)}\]

From our previous result, we know that the general solution is
\[y(t) = y_H(t) + y_p(t) \ t \in I\]
where \(y_H\) is the general solution to
\[p(t) \frac{d^2 y}{dt^2} + q(t) \frac{dy}{dt} + r(t) y = 0\]
and \(y_p\) is a particular solution to the given eq. Also, we know that
\[y_H(t) = c_1 y_1(t) + c_2 y_2(t) \ c_1, c_2 \in \mathbb{R}\]
where \(\{y_1, y_2\}\) is a basis for the set of solutions to the homogeneous eq.

Once we find \(\{y_1, y_2\}\), we find \(y\).

\begin{example}
Let's find a basis of the set of solutions for the eq.
\[\frac{d^2 y}{dt^2} - 2t \frac{dy}{dt} - 2y = 0\]

Here, \(p(t)=1\), \(q(t)=-2t\), \(r(t)=-2\) (polynomials).

We'll look for solutions in the form
\[y(t) = \sum_{n=0}^{\infty} a_n t^n \tag{32}\]
where the coefficients \(a_n\) must be determined.

\begin{recap}[Brief review on power series]
\begin{itemize}
\item Any series of the form \(\sum_{n=0}^\infty a_n (t - t_0)^n\) is called a power series about \(t_0\). Any power series has an interval of convergence. Specifically, there exists \(\rho > 0\) such that the series is convergent if \(|t - t_0| < \rho\) and the series diverges if \(|t - t_0| > \rho\). By the ratio test, we know that if \(|\frac{a_{n+1}}{a_n}| \to \lambda\) as \(n \to \infty\) for some \(\lambda > 0\), then \(\rho = \frac{1}{\lambda}\).
\item All power series can be integrated and differentiated term-by-term, and the resulting series have the same radius of convergence as the original series.
\end{itemize}
\end{recap}

Let's \(\rho > 0\) be the radius of convergence of (32). From now on we assume that \(|t| < \rho\). Observe that
\begin{align*}
\frac{dy}{dt} &= \sum_{n=1}^\infty n a_n t^{n-1} = \sum_{n=0}^\infty (n+1) a_{n+1} t^{n}\\
\frac{d^2 y}{dt^2} &= \sum_{n=2}^\infty n(n-1) a_n t^{n-2} = \sum_{n=0}^\infty n(n+1) a_{n+2} t^{n}
\end{align*}

Then, \(y(t)\) solves the eq. iff
\begin{align*}
0 &= \sum_{n=0}^\infty n(n-1) a_n t^{n-2} - 2t \sum_{n=0}^\infty n a_n t^{n-1} - 2 \sum_{n=0}^\infty a_n t^n \\
&= \sum_{n=0}^\infty n(n-1) a_n t^{n-2} - \sum_{n=0}^\infty 2n a_n t^n - \sum_{n=0}^\infty 2 a_n t^n \\
&= \underbrace{\sum_{k=-2}^\infty (k+2)(k+1) a_{k+2} t^k}_{k = n-2} - \sum_{n=0}^\infty 2n a_n t^n - \sum_{n=0}^\infty 2 a_n t^n \\
&= \sum_{n=0}^\infty (n+2)(n+1) a_{n+2} t^n - \sum_{n=0}^\infty 2n a_n t^n - \sum_{n=0}^\infty 2 a_n t^n \\
&= \sum_{n=0}^\infty \left( (n+2)(n+1) a_{n+2} - 2n a_n - 2 a_n \right) t^n 
\end{align*}

From here, we notice that we can look for the coeff. from
\[(n+2)(n+1) a_{n+2} - 2(n+1) a_n = 0, \ n = 0, 1, 2, \cdots\]

Thus,
\[a_{n+2} = \frac{2 a_n}{n+2} \ n = 0, 1, 2, \cdots\]

If \(a_{n+2} = \frac{2 a_n}{n+2}\) for \(n \geq 0\) then \(y(t) = \sum_{n=0}^\infty a_n t^n\) solves the eq.

Observe that \(a_0\) and \(a_1\) must be given.
\end{example}
