\section{Lecture 3 - 10.21}

\begin{recap}
Solution Methods for Some First-Order ODEs
\begin{itemize}
\item The integrating factor method for linear equations:
\[\frac{d\tilde{y}}{dt} = \tilde{b}(t) \tag{*}\]
(Idea: Transform the given equation into something in the form (*).)

\item Separable equations: \(\mu(y)\frac{dy}{dt} = g(t)\).
\end{itemize}
\end{recap}

\subsection{Exact Equations}

Consider the equation \(\frac{dy}{dt} = f(t, y)\). If we can transform it to
\[\frac{d\phi(t, y)}{dt} = 0 \tag{17},\]
then solutions are \(\phi(t, y) = c\) for some constant \(c \in \mathbb{R}\).

It's common to deal with equations written as
\[M(t, y) + N(t, y)\frac{dy}{dt} = 0 \tag{18}\]

\begin{note}
For \(\frac{dy}{dt} = f(t, y)\), we have \(N \equiv 1\) and \(M = -f(t, y)\).
\end{note}

Now, observe that
\[\frac{d\phi(t, y)}{dt} = \frac{\partial\phi}{\partial t} + \frac{\partial\phi}{\partial y}\frac{dy}{dt} \tag{\textcolor{red}{*}}\]

If we find \(\phi\) such that \(\frac{\partial\phi}{\partial t} = M\) and \(\frac{\partial\phi}{\partial y} = N\), then the equation (18) has the form (17), and so \(y\) is given by \(\phi(t, y) = c\) for \(c \in \mathbb{R}\).

\begin{remark}[on (\textcolor{red}{*})]
Consider a function \(\phi(t, u)\), \((t, u) \in \mathbb{R}^2\). Then, \(\phi(t, y(t))\) is a function of \(t\) only.

To compute \(\frac{d\phi(t, y(t))}{dt}\), we use the chain rule
\[\frac{d\phi(t, y(t))}{dt} = \frac{\partial\phi}{\partial t} \cdot \frac{dt}{dt} + \frac{\partial\phi}{\partial u} \cdot \frac{dy(t)}{dt}\]

Since \(\frac{dt}{dt} = 1\), this simplifies to
\[\frac{d\phi(t, y(t))}{dt} = \frac{\partial\phi}{\partial t} + \frac{\partial\phi}{\partial u} \cdot \frac{dy}{dt}\]

By an abuse of notation, we usually write \(u = y\).
\end{remark}

\begin{theorem}[Exactness Condition]
Let \(M\) and \(N\) be continuous functions with continuous first-order partial derivatives. Then, there exists a function \(\phi\) that satisfies the exactness condition if and only if
\[\frac{\partial M}{\partial y} = \frac{\partial N}{\partial t}\]
\end{theorem}

\begin{remark}
Proof is commented out!
\end{remark}

% \begin{proof}
% A function \(\phi\) satisfies \(\frac{\partial \phi}{\partial t} = M\) if and only if
% \[\phi(t, y) = \int M(t, y) dt + h(y)\]

% Also, \(\phi\) satisfies \(\frac{\partial \phi}{\partial y} = N\) if
% \[\frac{\partial \phi(t, y)}{\partial y} = \frac{\partial}{\partial y} \left( \int M(t, y) dt + h(y) \right) = N(t, y)\]

% That is
% \[N(t, y) = \frac{d}{dy} \int M(t, y) dt + \frac{dh}{dy}\]

% Thus, a function \(\phi\) with the exactness property exists if and only if \(\phi(t, y) = \int M(t, y) dt + h(y)\), where \(h(y)\) is such that
% \[\frac{dh}{dy} = N(t, y) - \frac{d}{dy} \int M(t, y) dt\]
% (Both sides are functions of \(y\) only.)

% Notice that \(h\) exists if and only if the right-hand side depends only on \(y\). That is, if and only if
% \[\frac{\partial}{\partial t} \left( N(t, y) - \frac{d}{dy} \int M(t, y) dt \right) = 0,\]
% which simplifies to \(\frac{\partial N}{\partial t} = \frac{\partial}{\partial t} \left( \frac{d}{dy} \int M(t, y) dt \right)\).

% Equivalently, if and only if
% \[\frac{\partial N}{\partial t}(t, y) - \frac{\partial}{\partial t} \int \frac{\partial M}{\partial y}(t, y) dt = \frac{\partial N}{\partial t}(t, y) - \frac{\partial M}{\partial y}(t, y) = 0\]
% \end{proof}

\begin{remark}[Differentiation Under the Integral Sign]
In the proof we used that \(\frac{\partial}{\partial y} \int M(t, y) dt = \int \frac{\partial}{\partial y} M(t, y) dt\). This holds because \(M\) is continuous and \(\frac{\partial M}{\partial y}\) is continuous as well. The proof of this is based on the Dominated Convergence Theorem (DCT):
\begin{itemize}
\item \(\{f_n(t)\}\), \(f_n(t) \to f(t)\) for every \(t\)
\item \(|f_n(t)| \leq g(t)\) for every \(t\)
\item \(\int_a^b g(t) dt\) exists
\end{itemize}

Then \(\lim_{n \to \infty} \int_a^b f_n(t) dt = \int_a^b \lim_{n \to \infty} f_n(t) dt = \int_a^b f(t) dt\).
\end{remark}

\begin{definition}[Exact Equation]
Any first-order ODE in the form \(M(t, y) + N(t, y)\frac{dy}{dt} = 0\) that satisfies \(\frac{\partial M}{\partial y} = \frac{\partial N}{\partial t}\) is called an \textbf{exact equation}.
\end{definition}

\subsection{Integrating Factor for Non-Exact Equations}
Consider an equation \(M(t, y) + N(t, y)\frac{dy}{dt} = 0\) which is not exact (that is, \(\frac{\partial M}{\partial y} \neq \frac{\partial N}{\partial t}\)).

Is there any function \(\mu(t, y)\) such that
\[\mu(t, y)\left( M(t, y) + N(t, y)\frac{dy}{dt} \right) = 0\]
is exact? In other words, is there any \(\mu(t, y)\) such that \(\frac{\partial \tilde{M}}{\partial y} = \frac{\partial \tilde{N}}{\partial t}?\)

(\(\tilde{M}(t, y) = \mu(t, y)M(t, y)\), \(\tilde{N}(t, y) = \mu(t, y)N(t, y)\).)

Recall the exactness condition \(\frac{\partial \tilde{M}}{\partial y} = \frac{\partial \tilde{N}}{\partial t}\) for \(\tilde{M} = \mu M\) and \(\tilde{N} = \mu N\), which gives
\[\frac{\partial \mu}{\partial y} M + \mu \frac{\partial M}{\partial y} = \frac{\partial \mu}{\partial t} N + \mu \frac{\partial N}{\partial t} \tag{23}\]

To simplify, we look for \(\mu\) (called an integrating factor) depending only on \(t\) (so \(\frac{\partial \mu}{\partial y} = 0\)). Equation (23) reduces to
\[\mu \frac{\partial M}{\partial y} = \frac{d\mu}{dt} N + \mu \frac{\partial N}{\partial t}\]

Equivalently,
\[\frac{d\mu(t)}{dt} N = \mu(t) \left( \frac{\partial N}{\partial t} - \frac{\partial M}{\partial y} \right)\]

If \(N\) never vanishes, then
\[\frac{d\mu(t)}{dt} = \frac{\mu(t)}{N} \left( \frac{\partial N}{\partial t} - \frac{\partial M}{\partial y} \right)\]
(The right-hand side is a function of \(t\) only.)

Notice that the previous equation makes sense if and only if \(\frac{1}{N} \left( \frac{\partial N}{\partial t} - \frac{\partial M}{\partial y} \right)\) depends on \(t\) only.

\subsection{\S2. The Existence and Uniqueness Theorem for First-Order ODEs}

Consider the initial value problem (IVP):
\[\begin{cases}
\frac{dy}{dt} = f(t, y) \\
y(t_0) = y_0
\end{cases}\]

\begin{example}
\[\begin{cases}
\frac{dy}{dt} = 1 + y^2 \\
y(0) = 0
\end{cases}\]

Here, \(f(t, y) = 1 + y^2\), \(t_0 = 0\), \(y_0 = 0\).

Notice that the ODE is separable. Let's look for the general solution:
\[\frac{dy}{dt} = 1 + y^2 \iff \frac{1}{1 + y^2} \frac{dy}{dt} = 1\]

Integration yields:
\[\int \frac{y'(t)}{1 + y(t)^2} dt = \int dt\]

Thus,
\[\arctan(y(t)) = t + c, \ c \in \mathbb{R}\]

Therefore,
\[y(t) = \tan(t + c), \ c \in \mathbb{R}\]

Additionally, to have \(y(0) = 0\), we need \(\tan(c) = 0\). We choose \(c = 0\), so
\[y(t) = \tan(t), \ t \in \left( -\frac{\pi}{2}, \frac{\pi}{2} \right)\]
\end{example}
