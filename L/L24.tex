\section{Lecture 24 - 01.06}

\begin{theorem}
Consider the regular Sturm-Liouville problem
{\color{blue}\begin{align*}
& \frac{d}{dt}\left(p\frac{dy}{dt}\right) - qy = -\lambda r y, \ \alpha < t < \beta \\
& B_\alpha[y] = a_1 y(\alpha) + b_1 \frac{dy}{dt}(\alpha) = 0 \\
& B_\beta[y] = a_2 y(\beta) + b_2 \frac{dy}{dt}(\beta) = 0
\end{align*}}
where \(p, p', q, r \in C[\alpha, \beta]\), \(p(t)>0, r(t)>0\) for all \(t \in [\alpha, \beta]\), \(|a_1| + |b_1| > 0\), \(|a_2| + |b_2| > 0\), and \(-\infty < \alpha < \beta < \infty\).
Then, the following is true:
\begin{enumerate}[a)]
\item {\color{blue}Eigenvalues exist and all of them are real.}
\item {\color{blue}Eigenvalues are simple, i.e., the eigenspace of each eigenvalue has dimension 1.}
\item {\color{blue}Eigenfunctions are real, apart from a multiplicative constant. Also, eigenfunctions with different eigenvalues are orthogonal with respect to the inner product \((u,v) = \int_{\alpha}^{\beta} u v \, dt\), for \(u,v\) real-valued.}
\item {\color{blue}The eigenvalues are countable, can be written as \(\lambda_0, \lambda_1, \cdots\) with \(\lambda_0 < \lambda_1 < \lambda_2 < \cdots\). Furthermore, \(\lim_{n\to\infty} \lambda_n = +\infty\). If \(y_n\) is an eigenfunction of \(\lambda_n\), then \(n\) is the number of zeros of \(y_n\) in \((\alpha, \beta)\). Additionally, the zeros of \(y_{n+1}\) interlace the zeros of \(y_n\).}
\item {\color{blue}If \(f \in C^1[\alpha, \beta]\), then the series \(\sum_{n=0}^{\infty} B_n y_n\) converges uniformly to \(f\) in any closed interval \(I \subset (\alpha, \beta)\), where \(B_n = \frac{(f, y_n)}{(y_n, y_n)}\), \(n=0, 1, \cdots\) (eigenfunction expansion of \(f\)).}
\end{enumerate}
\end{theorem}

\begin{remark}
\sout{Proof is commented out!}
\end{remark}

% \begin{proof}
% \begin{enumerate}[a)]
% \item \checkmark (part)
% \item \checkmark
% \item We have to prove that if \(\omega\) is an eigenfunction, then \(\omega = A \tilde{\omega}\) where \(A \in \mathbb{C}\) and \(\tilde{\omega}\) assumes real values.
% Assume that \(\omega\) is an eigenfunction with eigenvalue \(\lambda\), and write \(\omega = u + i v\) where \(u\) and \(v\) are real-valued.
% \begin{itemize}
% \item If \(v \equiv 0\), then \(\omega = u\). \checkmark
% \item If \(u \equiv 0\), then \(\omega = i v\). \checkmark
% \end{itemize}

% Assume that \(u \not\equiv 0\) and \(v \not\equiv 0\). It is easy to show that \(u, v\) are eigenfunctions with eigenvalue \(\lambda\). As \(\lambda\) is simple (from part (b)), we know that \(u\) and \(v\) are linearly dependent. So, \(u = \gamma v\) for some \(\gamma \in \mathbb{C}\) (since \(u,v\) are real, \(\gamma\) is real).
% Therefore, \(\omega = u + iv = \gamma v + iv = (\gamma + i)v = A \tilde{\omega}\), where \(A = \gamma+i\) and \(\tilde{\omega} = v\).

% To show that eigenfunctions (possibly complex-valued) are orthogonal, with respect to \((u,v)_c = \int_{\alpha}^{\beta} u \bar{v} \, dt\), it is enough to prove that real-valued eigenfunctions are orthogonal with respect to \((u,v) = \int_{\alpha}^{\beta} u v \, dt\).
% As \(L[y] = -\frac{d}{dt}(p\frac{dy}{dt}) - qy\) is self-adjoint in the sense that \((L[y], z) = (y, L[z])\) for every \(y,z \in C^2[\alpha, \beta]\) satisfying the boundary conditions \(B_\alpha[y]=B_\alpha[z]=B_\beta[y]=B_\beta[z]=0\).
% If \(y,z\) are eigenfunctions with eigenvalues \(\lambda\) and \(\nu\), then \((\lambda y, z) = (y, \nu z) \implies \lambda(y,z) = \nu(y,z) \implies (\lambda - \nu)(y,z) = 0\).
% If \(\lambda \neq \nu\), then \((y,z) = 0\).

% \item[d) e)] Are out of the scope here (Coddington-Levinson).
% \end{enumerate}
% \end{proof}

\begin{example}
Consider again the problem
{\color{blue}\[\begin{cases}
\frac{d^2y}{dt^2} + \lambda y = 0, & 0 < t < L \\
y(0)=y(L)=0
\end{cases}\]}

We know that the eigenvalues are {\color{blue}\(\lambda_n = \left(\frac{n\pi}{L}\right)^2\)} and the corresponding eigenfunctions are {\color{blue}\(y_n(t) = \sin\left(\frac{n\pi t}{L}\right), \ t \in [0,L]\), \(n=1, 2, \cdots\).}
Now we know that these are \textit{all} the eigenvalues of this problem.
Assume, for simplicity, that \(L=\pi\), so \(\lambda_n = n^2\) and \(y_n(t) = \sin(nt)\), \(n=1, 2, \cdots\).
If we relabel the eigenvalues and eigenfunctions as \(\tilde{\lambda}_n = (n+1)^2\), \(\tilde{y}_n(t) = \sin((n+1)t)\), \(n=0, 1, \cdots\).

Observe that {\color{blue}\(\tilde{y}_n\) has \(n\) zeros in \((0,\pi)\). Also, the zeros of \(\tilde{y}_{n+1}\) interlace those of \(\tilde{y}_n\).}
Additionally, \(\lim_{n\to\infty} \tilde{\lambda}_n = +\infty\).

Consider the function {\color{blue}\(f(t) = t(\pi - t)\), \(t \in [0, \pi]\).}

Let's compute the eigenfunction expansion \(\sum_{n=1}^{\infty} B_n y_n\).

Let's compute the coeff. \(B_n = \frac{(f, y_n)}{(y_n, y_n)}\), \(n=1, 2, \cdots\).
\begin{itemize}
\item \((y_n, y_n) = \int_0^\pi y_n^2 \, dt = \int_0^\pi \sin^2(nt) \, dt = \frac{\pi}{2}\)
\item \((f, y_n) = \int_0^\pi t(\pi-t)\sin(nt) \, dt = \cdots = \begin{cases} 0 & \text{if } n=2k \\ \frac{4}{n^3} & \text{if } n=2k+1 \end{cases}\)
\end{itemize}

Thus, {\color{blue}\(B_n = \begin{cases} 0 & \text{if } n=2k \\ \frac{8}{\pi n^3} & \text{if } n=2k+1 \end{cases}\)}.
Therefore, the eigenfunction expansion of \(f\) with respect to the eigenfunctions of the problem is
{\color{blue}\[\underbrace{\sum_{k=0}^{\infty} \frac{8}{\pi(2k+1)^3} \sin((2k+1)t)}_{=S}.\]}

Let's prove, in this special case, part (e) of the theorem, that is, that \(f\) is the uniform limit of the series in every closed interval \(I \subset (0, \pi)\).

First, notice that \(S\) converges uniformly to some function \(g\) in \([0,\pi]\) since:
\begin{enumerate}[a)]
\item {\color{blue}\(\left| \frac{8}{\pi(2k+1)^3} \sin((2k+1)t) \right| \leqslant \frac{8}{\pi(2k+1)^3}, \ \forall t \in [0, \pi], \forall k \in \mathbb{N}_0\).}
\item {\color{blue}\(\sum_{k=0}^{\infty} \frac{8}{\pi(2k+1)^3} \leqslant \frac{8}{\pi} \sum_{n=1}^{\infty} \frac{1}{n^3} < \infty\).}
\end{enumerate}

{\color{blue}Then, by the Weierstrass M-test, we know that \(S\) converges uniformly to some \(g\).}

Also, as each term of \(S\) is continuous in \([0,\pi]\), and the convergence is uniform, we have that \(g\) is continuous in \([0,\pi]\).

\(\implies\) Assume for a moment that \(\frac{dg}{dt}\) exists in \([0, \pi]\) and \(\frac{dg}{dt} = \sum_{k=0}^{\infty} \frac{d}{dt} \left( \frac{8 \sin((2k+1)t)}{\pi (2k+1)^3} \right)\).

\(\implies\) If we prove that
\[\frac{dg}{dt}(t) \stackrel{?}{=} \frac{df}{dt}(t) = \pi - 2t \ \text{for } t \in [0, \pi],\]
then \(g+c = f\) for some constant \(c \in \mathbb{R}\).

If we prove this, and notice that \(f(0)=0\), \(g(0)=0 \implies c=0\), so \(f=g\).
\end{example}
