\section{Tutorial 1 - 10.17}

\subsection{Linear equation: the integrating factor method}

{\color{blue}\(\underbrace{\mu(t) (\frac{dy}{dt} + a(t)y)}_{\frac{d}{dt}(\mu(t) y)} = b(t) \mu(t)\), \(a, b\) are continuous functions in \(\mathbb{R}\).
\(\mu(t) = e^{\int a(t)dt}\), \(\mu\) can be any solution to \(\frac{d\mu}{dt} = a(t)\mu\).
\(y(t) = e^{-\int a(t) dt} (\int b(t)e^{\int a(t) dt} dt + c)\), \(C \in \mathbb{R}, t \in \mathbb{R}\). (General solution to \(\frac{dy}{dt} + a(t)y = b(t)\).)}

\begin{xca}
Solve the initial-value problem
\[\begin{cases}
\frac{dy}{dt} + ty = 2t \\
y(0) = 1
\end{cases}\]
\end{xca}
\begin{proof}[Solution]
Observe that the given equ. is linear, so it can be written as
\[\frac{dy}{dt} + a(t)y = b(t) \ \text{where} \ a(t) = t, \ b(t) = 2t\]
with \(a(t), b(t)\) continuous in \(\mathbb{R}\).

So the general solution to the eq. is given by
\begin{align*}
y(t) &= e^{-\int a(t) dt} \left( \int b(t) e^{\int a(t) dt} dt + C \right) \\
&= e^{-\int t dt} \left( \int 2t e^{\int t dt} dt + C \right) \\
&= e^{-t^2/2} \left( \int 2t e^{t^2/2} dt + C \right) \\
&= e^{-t^2/2} \left( 2e^{t^2/2} + C \right) \\
&= {\color{blue}2 + Ce^{-t^2/2}, \ C \in \mathbb{R}, \ t \in \mathbb{R}}
\end{align*}

\(y(0) = 2 + C e^0 = 2 + C = 1 \iff {\color{blue}C = -1}\).

So, {\color{blue}\(y(t) = 2 - e^{-t^2/2}, \ t \in \mathbb{R}\).}
\end{proof}

\subsection{Separable equations}

{\color{blue}Given an ODE of first-order, \(\frac{dy}{dt} = f(t, y)\), we say that the equation is separable if it can be written as
\[\frac{dy}{dt} = \frac{g(t)}{h(y)}\]}

Observe that the equ. is equivalent to
\begin{align*}
h(y) \frac{dy}{dt} &= g(t) \\
h(y(t)) y'(t) &= g(t)
\end{align*}

Integrate:
\begin{align*}
\int h(y(t)) y'(t) dt &= \int g(t) dt \\
{\color{blue}\int h(y) dy} &= {\color{blue}\int g(t) dt},
\end{align*}
where \(y = y(t)\).

By solving, if possible, the integral in the LHS, we find a formula for \(y(t)\) which sometimes gives us \(y(t)\) in an implicit way.

In the context of ODEs. a first-order linear equation is one of the form:
\[\underbrace{\left[ \frac{dy}{dt} + a(t)y \right]}_{\text{differential operator}} = b(t)\]
or more generally:
\[c(t)\frac{dy}{dt} + a(t)y = b(t).\]
    
Observe that \(f(t, y) = b(t) - a(t)y\). We have
\[f(t, \alpha y + \beta z) \neq \alpha f(t, y) + \beta f(t, z).\]

\begin{xca}
Solve the initial-value problem
\[\begin{cases}
e^y \frac{dy}{dt} = t \\
y(1) = 1
\end{cases}\]
\end{xca}
\begin{proof}[Solution]
Notice that the equ. can be written as
\[\frac{dy}{dt} = f(t, y) \ \text{where} \ f(t, y) = \frac{g(t)}{h(y)}, \ g(t) = t, \ h(y) = e^y.\]

Observe that the ODE is nonlinear. However, it's separable.
\begin{align*}
\int e^{y(t)} y'(t) dt &= \int t dt \\
e^{y(t)} &= t^2/2 + C \\
y(t) &= {\color{blue}\ln\left( \frac{t^2}{2} + C \right)}
\end{align*}

Notice that
\begin{itemize}
\item if \(C > 0\), then \(\frac{t^2}{2} + C > 0\) for all \(t \in \mathbb{R}\);
\item if \(C \leq 0\), then \(\frac{t^2}{2} + C > 0 \iff |t| > \sqrt{-2C} \iff t \in (-\infty, -\sqrt{-2C}) \cup (\sqrt{-2C}, +\infty)\).
\end{itemize}

{\color{blue}We have
\begin{align*}
y(1) = 1 \iff \ln\left( \frac{1}{2} + C \right) &= 1 \\
\frac{1}{2} + C &= e \\
C = e - \frac{1}{2} &> 0
\end{align*}

Therefore, \(y(t) = \ln\left( \frac{t^2}{2} + e - \frac{1}{2} \right), \ t \in \mathbb{R}\).}
\end{proof}
