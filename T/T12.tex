\section{Tutorial 12 - 01.09}

\subsection{Example (cont.)}

Recall that we are working with the regular Sturm-Liouville problem
{\color{blue}\[\begin{cases}
\frac{d^2y}{dt^2} + \lambda y = 0 & 0 < t < 1 \\
\frac{dy}{dt}(0) = 0, \ y(1) + \frac{dy}{dt}(1) = 0
\end{cases}\]}

\begin{enumerate}
\item {\color{blue}If \(\lambda \le 0\) then the unique solution is \(y \equiv 0\).}
\item {\color{blue}If \(\lambda > 0\) and \(\sqrt{\lambda} = \frac{\pi}{2} + n\pi\) for some \(n \in \mathbb{N}\) then the unique solution is \(y \equiv 0\). Also, if \(\sqrt{\lambda} \neq \frac{\pi}{2} + n\pi\) for every \(n \in \mathbb{N}\), then if \(\lambda\) solves \(1 - \sqrt{\lambda} \tan(\sqrt{\lambda}) = 0\), then the problem admits a nontrivial solution given by \(y(t) = \cos(\sqrt{\lambda}t)\).}
\end{enumerate}

Then, we notice that there is a countable number of \(\lambda\), given by:
{\color{blue}\[\begin{cases}
\lambda_0 & \text{is the root of } 1 = \sqrt{x} \tan(\sqrt{x}) \text{ in } (0, \frac{\pi}{2}) \\
\lambda_n & \text{is the root of } 1 = \sqrt{x} \tan(\sqrt{x}) \text{ in } (\frac{\pi}{2} + (n-1)\pi, \frac{\pi}{2} + n\pi), \, n \ge 1.
\end{cases}\]}

{\color{blue}The eigenfunctions are \(y_n(t) = \cos(\sqrt{\lambda_n} t)\), \(t \in [0, 1]\), \(n \ge 0\).}

\sout{Let's now show that the eigenfunctions with distinct eigenvalues are orthogonal w.r.t. \((y, z) = \int_0^1 y z \, dt\).
Let \(n \neq m\), \(n, m \in \mathbb{N}\).}
\begin{align*}
(\cos(\sqrt{\lambda_n} t), \cos(\sqrt{\lambda_m} t)) &= \int_0^1 \cos(\sqrt{\lambda_n} t) \cos(\sqrt{\lambda_m} t) \, dt \\
&= \frac{\sqrt{\lambda_n} \sin(\sqrt{\lambda_n}) \cos(\sqrt{\lambda_m}) - \sqrt{\lambda_m} \sin(\sqrt{\lambda_m}) \cos(\sqrt{\lambda_n})}{\lambda_n - \lambda_m}
\end{align*}

Since \(\frac{\sqrt{\lambda_k} \sin(\sqrt{\lambda_k}) \cos(\sqrt{\lambda_k})}{\cos(\sqrt{\lambda_k}) \cos(\sqrt{\lambda_k})} = \sqrt{\lambda_k} \tan(\sqrt{\lambda_k}) = 1\) (from boundary condition), we have
{\color{blue}\[= \frac{\cos(\sqrt{\lambda_n}) \cos(\sqrt{\lambda_m})}{\lambda_n - \lambda_m} \left( \overbrace{\sqrt{\lambda_n} \tan(\sqrt{\lambda_n})}^{=1} - \overbrace{\sqrt{\lambda_m} \tan(\sqrt{\lambda_m})}^{=1} \right) = 0.\]}

{\color{blue}Assume that \(f \in C^1[0, 1]\), then the eigenfunction expansion of \(f\) is}
{\color{blue}\[\sum_{n=0}^\infty B_n y_n \ \text{where } B_n = \frac{(f, y_n)}{(y_n, y_n)}.\]}

Then,
{\color{blue}\[f(t) = \sum_{n=0}^\infty \left( \int_0^1 f(s) \cos(\sqrt{\lambda_n} s) \, ds \right) \cos(\sqrt{\lambda_n} t), \ \text{uniformly}\]}
{\color{blue}for \(t \in I \subset (0, 1)\), \(I\) closed interval.}

\subsection{Example}

Consider the problem
{\color{blue}\[\begin{cases}
t^2 \frac{d^2y}{dt^2} = \lambda \left( t \frac{dy}{dt} - y \right) & 1 < t < 2 \\
y(1) = 0, \, y(2) = 0
\end{cases}\]}

Observe that this is not a Sturm-Liouville problem (\(\lambda\) appears in the coeff. of \(\frac{dy}{dt}\)). Let's look for \(\lambda\) to have nontrivial solutions to the problem.

Assume first that \(\lambda \in \mathbb{R}\). Notice that the eq. can be written as
{\color{blue}\[t^2 \frac{d^2y}{dt^2} - \lambda t \frac{dy}{dt} + \lambda y = 0, \ 1 < t < 2\]}

{\color{blue}This is the Euler's eq. \(t^2 \frac{d^2y}{dt^2} + \alpha t \frac{dy}{dt} + \beta y = 0\), for \(\alpha = -\lambda\) and \(\beta = \lambda\).}

To find a basis for the set of solutions, we need to analyze the sign of \((\alpha-1)^2 - 4\beta\).

In our case,
{\color{blue}\[(\alpha-1)^2 - 4\beta = (-\lambda-1)^2 - 4\lambda = (\lambda+1)^2 - 4\lambda = (\lambda-1)^2 \ge 0.\]}

{\color{blue}If \(\lambda \neq 1\), then \((\alpha-1)^2 - 4\beta > 0\).}
\sout{Then, the general solution to the eq. is} {\color{blue}\(y(t) = A t^{r_1} + B t^{r_2}\)} \sout{where \(r_1\) and \(r_2\) are two different solutions to \(r^2 - (\lambda+1)r + \lambda = 0\) (review the Euler's eq.).
Also, \(y(1) = A + B = 0\) iff \(A = -B\), and}
\[y(2) = A 2^{r_1} - A 2^{r_2} = A \underbrace{\left( 2^{r_1} - 2^{r_2} \right)}_{\neq 0} = 0 \iff A = 0.\]

The unique sol. is {\color{blue}\(y \equiv 0\).}

Assume that \(\lambda = 1\). Then \((\alpha-1)^2 - 4\beta = 0\). The general solution to the eq. is
{\color{blue}\[y(t) = A t^r + B t^r \ln t, \ A, B \in \mathbb{R}.\]}

Also, \(y(1) = A = 0\), and
\[y(2) = B \underbrace{2^r \ln 2}_{\neq 0} = 0 \iff B = 0.\]

Then, the unique sol. is {\color{blue}\(y \equiv 0\).}

{\color{blue}Therefore, the problem does not admit real eigenvalues.}
Now consider \(\lambda \in \mathbb{C}, \lambda \notin \mathbb{R}\).
