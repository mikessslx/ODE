\section{Tutorial 8 - 12.05}

\subsection{Exercise}

\begin{xca}
Find the general solution to
\[\frac{d^2y}{dt^2} - 2\frac{dy}{dt} + y = t\]
\end{xca}
\begin{proof}[Solution]
As the equation is linear with continuous coefficients and continuous right-hand side in \(\mathbb{R}\), the general solution is \(y(t) = y_H(t) + y_p(t), \ t \in \mathbb{R}\), where \(y_H\) is the general solution to
\[\frac{d^2y}{dt^2} - 2 \frac{dy}{dt} + y = 0\]
and \(y_p\) is a particular solution.

As the homogeneous equation has constant coefficients, we have a solution method to find \(y_H\).
The characteristic polynomial is \(p(\lambda) = \lambda^2 - 2\lambda + 1 = (\lambda - 1)^2\), and it has \(\lambda = 1\) as root with multiplicity 2.
Then, the functions \(y_1(t) = e^t, \ y_2(t) = te^t, \ t \in \mathbb{R}\), define a basis of solutions.
From this, we have \(y_H(t) = c_1 y_1 + c_2 y_2, \ c_1, c_2 \in \mathbb{R}\).

\begin{remark}
\[W(y_1, y_2)(t) = \det \begin{pmatrix} y_1(t) & y_2(t) \\ \frac{dy_1(t)}{dt} & \frac{dy_2(t)}{dt} \end{pmatrix} = \det \begin{pmatrix} e^t & te^t \\ e^t & (1+t)e^t \end{pmatrix} = ((1+t) - t)e^{2t} = e^{2t} \neq 0 \ \forall t\]

So \(e^t\) and \(te^t\) are linearly independent in \(\mathbb{R}\).
\end{remark}

Let's look for a particular solution \(y_p\) in the form \(y_p(t) = at + b\).

\begin{remark}
Consider the system \(\frac{dY}{dt} = AY\), \(Y^{(1)} = \binom{y_1^{(1)}}{y_2^{(1)}}\), \(Y^{(2)} = \binom{y_1^{(2)}}{y_2^{(2)}}\).
\(Y^{(1)}, Y^{(2)}\) are l.i. iff \(Y^{(1)}(t_0), Y^{(2)}(t_0)\) are l.i.

Using our functions: \(Y^{(1)}(t) = \binom{e^t}{e^t}\), \(Y^{(2)}(t) = \binom{te^t}{(1+t)e^t}\).
If \(t=0\), \(Y^{(1)}(0) = \binom{1}{1}\), \(Y^{(2)}(0) = \binom{0}{1}\), which are l.i.
\end{remark}

Observe that \(y_p(t) = at + b, \frac{dy_p}{dt} = a, \frac{d^2y_p}{dt^2} = 0\).

Then,
\begin{align*}
\frac{d^2y_p}{dt^2} - 2\frac{dy_p}{dt} + y_p = t &\iff -2a + at + b = t \iff (a-1)t + (b-2a) = 0 \\
&\iff \begin{cases} a-1=0 \\ b-2a=0 \end{cases} \iff \begin{cases} a=1 \\ b=2 \end{cases}
\end{align*}

A particular solution is \(y_p(t) = t + 2\).

Then, \(y(t) = c_1 e^t + c_2 te^t + t + 2, \ t \in \mathbb{R}, \ c_1, c_2 \in \mathbb{R}\).
\end{proof}

\begin{xca}
look for a particular solution using the method of variation of parameters.
\end{xca}

\subsection{Example}

\begin{example}[Mechanical vibrations]
Consider the problem of describing the motion of an object of mass \(m\) which is attached to a spring suspended from a vertical rigid wall, as in the figure.
An elastic spring has the property that if it is compressed or stretched by \(\Delta l\) then it exerts a restoring force proportional to \(\Delta l\): \(k \Delta l\), where \(k\) is the spring-constant. At rest, if no external forces are present, \(k \Delta l = mg\) where \(g\) is the acceleration of gravity.

It can be shown that \(y\) is governed by
\[m \frac{d^2y}{dt^2} + c \frac{dy}{dt} + ky = F(t)\]

\begin{itemize}
\item \(m\): mass of the object
\item \(c\): constant related to the medium (\(c \ge 0\))
\item \(k\): spring-constant (\(k > 0\))
\item \(F\): external force
\end{itemize}

We'll analyze these cases:
\begin{itemize}
\item Free vibrations (\(c=0, F=0\))
\item Damped free vibrations (\(c \neq 0, F=0\))
\item Damped forced vibrations (\(c \neq 0, F \neq 0\))
\item Forced free vibrations (\(c=0, F \neq 0\))
    
\(\rightsquigarrow\) A special case models resonance. (Tacoma Bridge (1940))
\end{itemize}

\end{example}
