\section{Tutorial 2 - 10.24}

\subsection{Review}

Consider the equation
\[M(t, y) + N(t, y)\frac{dy}{dt} = 0 \ \left( \frac{d\phi(t, y(t))}{dt} = 0 \right)\]

{\color{blue}\begin{theorem}
If \(M\) and \(N\) are continuous functions with continuous partial derivatives in \(R = (a, b) \times (c, d)\), then there exists \(\phi\) such that \(\frac{\partial \phi}{\partial t} = M\) and \(\frac{\partial \phi}{\partial y} = N\) if and only if
\[\frac{\partial M}{\partial y} = \frac{\partial N}{\partial t}.\]

In that case the general solution to the equation is given by \(\phi(t, y) = c\) for \(c \in \mathbb{R}\).
\end{theorem}}

\subsection{Exercise}

\begin{xca}
Solve the equation
\[2y + e^t + (2t + \sin(y))\frac{dy}{dt} = 0\]
\end{xca}
\begin{proof}[Solution]
Notice that the equation is nonlinear and non-separable. Also, let \(M(t, y) = 2y + e^t\) and \(N(t, y) = 2t + \sin(y)\). These functions are continuous with continuous partial derivatives in \(\mathbb{R}^2\).

We have
\[\frac{\partial M}{\partial y}(t, y) = 2 \ \text{and} \ \frac{\partial N}{\partial t}(t, y) = 2 \ \text{for every } (t, y) \in \mathbb{R}^2.\]

We know that there exists \(\phi\) such that \(\frac{\partial \phi}{\partial y} = N\) and \(\frac{\partial \phi}{\partial t} = M\).

Then,
\begin{align*}
\frac{\partial \phi}{\partial t}(t, y) = M(t, y) &\implies \frac{\partial \phi}{\partial t}(t, y) = 2y + e^t \\
&\implies \phi(t, y) = 2y \int dt + \int e^t dt + h(y) \\
&\implies \phi(t, y) = 2yt + e^t + h(y), \\
\frac{\partial \phi}{\partial y}(t, y) = N(t, y) &\implies \frac{\partial}{\partial y} \left( 2yt + e^t + h(y) \right) = 2t + \sin(y) \\
&\implies 2t + h'(y) = 2t + \sin(y) \\
&\implies h'(y) = \sin(y).
\end{align*}

Then we choose \(h(y) = -\cos(y)\). Therefore, \(\phi(t, y) = 2yt + e^t - \cos(y)\).

{\color{blue}Finally, the solution is given implicitly by \(2yt + e^t - \cos(y) = C, \ C \in \mathbb{R}\).}
\end{proof}

\begin{xca}
Solve the equation
\[y^2 + 4ye^t + 2(y+e^t)\frac{dy}{dt} = 0\]
\end{xca}
\begin{proof}[Solution]
As before, we can check that we are under the assumptions of the theorem for \(M(t, y) = y^2 + 4ye^t\) and \(N(t, y) = 2(y+e^t)\), except the condition \(\frac{\partial M}{\partial y} = \frac{\partial N}{\partial t}\) since
\[\frac{\partial M}{\partial y}(t, y) = 2y + 4e^t \ \text{and} \ \frac{\partial N}{\partial t}(t, y) = 2e^t, \ (t, y) \in \mathbb{R}^2.\]

We look for some integrating factor \(\mu\) to transform the eq. into an exact eq.
\[\underbrace{\mu(t, y)(y^2 + 4ye^t)}_{\tilde{M}(t, y)} + \underbrace{\mu(t, y)2(y+e^t)}_{\tilde{N}(t, y)}\frac{dy}{dt} = 0\]

Recall that we deduced that there exists an integrating factor \(\mu = \mu(t)\) if and only if
\[\frac{1}{N} \left( \frac{\partial M}{\partial y} - \frac{\partial N}{\partial t} \right)\]
depends only on \(t\).

We have,
\[\frac{1}{2(y+e^t)} (2y + 4e^t - 2e^t) = \frac{2y + 2e^t}{2(y+e^t)} = 1\]

Also, we know that \(\mu\) must be a solution to
\[\frac{d\mu}{dt} = \frac{1}{N} \left( \frac{\partial M}{\partial y} - \frac{\partial N}{\partial t} \right) \mu \iff \frac{d\mu}{dt} = \mu\]

{\color{blue}We choose, \(\mu(t) = e^t\).}

From now on, the solution follows as in the previous example:

We look for \(\phi\) such that \(\frac{\partial \phi}{\partial t} = \mu(t)M(t, y)\) and \(\frac{\partial \phi}{\partial y} = \mu(t)N(t, y)\).

After the computations, we have \(\phi(t, y) = y^2e^t + 2ye^{2t}\) (exercise).

{\color{blue}The solution is given by \(y^2e^t + 2ye^{2t} = c, \ c \in \mathbb{R}\).}
\end{proof}

\begin{xca}
Find the constant \(a \in \mathbb{R}\) such that the equation
\[\frac{1}{t^2} + \frac{1}{y^2} + \frac{at+1}{y^3} \frac{dy}{dt} = 0\]
is exact, and then solve the equation.
\end{xca}
\begin{proof}[Solution]
Let \(M(t, y) = \frac{1}{t^2} + \frac{1}{y^2}\) and \(N(t, y) = \frac{at+1}{y^3}\). These functions are continuous and have continuous partial derivatives in every \(R = (a, b) \times (c, d)\) not containing the origin \((t, y) = (0, 0)\).

Let \(R\) a rectangle as before. We'll find \(a\) such that
\[\frac{\partial M}{\partial y} = \frac{\partial N}{\partial t} \rightsquigarrow a = -2. \ (\text{exercise})\]

{\color{blue}After solving the eq., we find the solution given by \(-\frac{1}{t} + \frac{1}{y^2}(t-1) = c, \ c \in \mathbb{R}\).}
\end{proof}

\begin{remark}
Suppose we also have the initial condition \(y(2) = 1\).

From this, \(c = \frac{1}{2}\). {\color{blue}The solution is \(y^2 = \frac{t-1}{2+t} 2t\).}

{\color{blue}This makes sense iff \(\frac{t-1}{2+t} 2t \geq 0\). From here, we find the domain (exercise).}
\end{remark}
