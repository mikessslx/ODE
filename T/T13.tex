\section{Tutorial 13 - 01.13}

\subsection{Theorem (cont.)}

\begin{theorem}
We are working with
{\color{blue}\begin{align*}
\frac{d}{dt} \left( p \frac{dy}{dt} \right) - qy &= -\mu r y + f \quad \alpha < t < \beta \tag{40} \\
B_\alpha[y] = B_\beta[y] &= 0 \tag{41}
\end{align*}}

{\color{blue}Only one of these are true:}
\begin{enumerate}
\item[(a)] {\color{blue}The nonhom. prob. admits a unique sol. for every \(f \in C[\alpha, \beta]\).}
\item[(b)] {\color{blue}The associated hom. prob. has non trivial solutions.}
\end{enumerate}
\end{theorem}

\begin{remark}
\sout{Proof is commented out!}
\end{remark}

% \begin{proof}[Proof (under additional assumptions)]
% Sketch from yesterday.

% A function \(y \in C^2(\alpha, \beta) \cap C^1[\alpha, \beta]\) satisfies (40) in \(I \subset (\alpha, \beta)\) and (41) iff \(y\) satisfies (41) and
% \[\sum_{n=1}^\infty ((\lambda_n - \mu)B_n + C_n) y_n = 0 \quad \text{in } I,\]
% where
% \[\frac{f}{r} = \sum_{n=1}^\infty C_n y_n \text{ in } I. \quad \text{Recall } y = \sum_{n=0}^\infty B_n y_n.\]

% If \((\lambda_n - \mu)B_n + C_n = 0\) for all \(n \ge 0\) then \(\sum_{n=1}^\infty ((\lambda_n - \mu)B_n + C_n) y_n = 0\) in \(I\).

% Reciprocally, if \(\sum_{n=0}^\infty (\lambda_n - \mu)B_n + C_n) y_n = 0\) in every \(I \subset (\alpha, \beta)\), \(I\) closed, then
% \begin{align*}
% 0 &= \int_\alpha^\beta r y_m \sum_{n=0}^\infty ((\lambda_n - \mu)B_m + C_m) y_n \, dt = \sum_{n=0}^\infty ((\lambda_n - \mu)B_n + C_n) \underbrace{\int_\alpha^\beta r y_n y_m \, dt}_{\begin{cases} 1 & \text{if } n=m \\ 0 & \text{if } n \neq m \end{cases}} \\
% &= (\lambda_m - \mu)B_m + C_m \quad \text{for all } m \ge 0.
% \end{align*}

% Assume that \(\mu \neq \lambda_n\) for all \(n \ge 0\).

% If \(B_n = \frac{C_n}{\mu - \lambda_n}\), \(n \ge 0\), then \(y = \sum_{n=0}^\infty B_n y_n\) solves the eq. in \((\alpha, \beta)\). Also,
% \begin{align*}
% B_\alpha[y] &= a_1 y(\alpha) + b_1 \frac{dy}{dt}(\alpha) = a_1 \sum_{n=0}^\infty B_n y_n(\alpha) + b_1 \sum_{n=0}^\infty B_n \frac{dy_n}{dt}(\alpha) \\
% &= \sum_{n=0}^\infty B_n \underbrace{\left( a_1 y_n(\alpha) + b_1 \frac{dy_n}{dt}(\alpha) \right)}_{=0} = 0, \text{ similarly } B_\beta[y] = 0.
% \end{align*}

% Then, (a) is true and (b) fails.

% Assume that \(\mu = \lambda_m\) for some \(m \ge 0\).
% The eq. \((\lambda_m - \mu)B_m + C_m = 0\) reduces to \(C_m = 0\).
% If \(C_m = 0\) then \(y = \sum_{n=0}^\infty B_n y_n\) solves the nonhom. problem where
% \[B_n = \frac{C_n}{\mu - \lambda_n} \text{ if } n \neq m, \text{ and } B_m \text{ is any number}.\]

% Then (a) fails because the nonhom. prob. has infinitely many solutions, and (b) is true.

% Finally, assume that \(C_m \neq 0\). Then \((\lambda_m + \mu)B_m + C_m = 0\) does not admit any solution. Then, there is no solution to the eq. in \((\alpha, \beta)\).
% Then, (a) fails because the problem has no solutions, and (b) is true.
% \end{proof}

\subsection{Final Review}

\paragraph{\S1. ODEs. Some Sol. Meth. For First-Order Eqs.}
\begin{itemize}
\item {\color{blue}Classification}: order, scalar eqs / system of eqs, linear and nonlinear eqs.
\item {\color{blue}Sol. meth.}: {\color{blue}meth. of integ. factor} for linear eqs., {\color{blue}direct integ.} for separable eqs., sol. meth. for {\color{blue}exact eqs.} and {\color{blue}meth. of integ. factors} for non-exact eqs.
\end{itemize}

\paragraph{\S2. The Existence and Uniq. Thm. For First-Order Eqs.}
\begin{align*}
\begin{cases} \frac{dy}{dt} = f(t, y) \\ y(t_0) = y_0 \end{cases} \quad
\begin{aligned}
&{\color{blue}\text{Picard-Lindelof } (\exists !)} \\
&{\color{blue}\text{Cauchy-Peano } (\exists)} \\
&{\color{blue}\text{Picard iterates.}}
\end{aligned}
\end{align*}

\begin{itemize}
\item Continuation of sol.
\item Linear eqs. with cont. coeff. have global sol.
\end{itemize}

\paragraph{\S3. Systems of First-Order Eqs ($n$th-order Eqs)}
\begin{itemize}
\item The results from \S 2 extend to the context of systems and eqs (as they have associated systems).
\item {\color{blue}Linear systems}, \(\frac{dy}{dt} = A(t)y + b(t)\). Structure of the set of solutions. Solution matrix. {\color{blue}Meth. of variation of parameters}.
\item {\color{blue}Wronskian}.
\end{itemize}

\paragraph{\S4. Linear Systems with Const. Coeff.}
\[\frac{dy}{dt} = Ay + b\]
\begin{itemize}
\item Basis of sol. to \(\frac{dy}{dt} = Ay\) is related to the {\color{blue}eigenval. and eigenvectors} of \(A\).
\item {\color{blue}Exponential matrix}.
\end{itemize}

\paragraph{\S5. Sol. Meth. for Second-Order Linear Eqs.}
\begin{itemize}
\item Const. coeff.: {\color{blue}Roots of the char. polynomial} and basis.
\item Non-const. coeff.: {\color{blue}Series sol. meth.}
\end{itemize}

\paragraph{\S6. Sturm-Liouville Theory.}
\begin{itemize}
\item {\color{blue}Main thm}.
\item {\color{blue}Fredholm alternative}.
\end{itemize}
