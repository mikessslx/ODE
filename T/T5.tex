\section{Tutorial 5 - 11.14}

\subsection{Exercises on Systems of ODEs}

\begin{xca}
Consider the IVP
\[\begin{cases}
\frac{dy_1}{dt} = y_2 \\
\frac{dy_2}{dt} = -y_1 - 2y_2 \\
y_1(0) = 1, y_2(0) = 1
\end{cases}\]
    
\begin{enumerate}[a)]
\item Prove that \(y^{(1)}(t) = \begin{pmatrix} e^{-t} \\ -e^{-t} \end{pmatrix}\) and \(y^{(2)}(t) = \begin{pmatrix} te^{-t} \\ (1-t)e^{-t} \end{pmatrix}\) form a basis of solutions to the associated homogeneous system.
\item Find the solution to the IVP.
\end{enumerate}
\end{xca}
\begin{proof}[Solution]
\begin{enumerate}[a)]
\item
Notice that the homogeneous system associated to the IVP is
\[\begin{cases} \frac{dy_1}{dt} = y_2 \\ \frac{dy_2}{dt} = -y_1 - 2y_2 \end{cases}, \text{ that is } \frac{dy}{dt} = Ay \text{ where } y = \begin{pmatrix} y_1 \\ y_2 \end{pmatrix} \text{ and } A = \begin{pmatrix} 0 & 1 \\ -1 & -2 \end{pmatrix}\]

This is a linear system with constant coefficients, thus the set of solutions is a vector space of dimension 2.

Let's check that \(y^{(1)}\) and \(y^{(2)}\) are solutions.
\begin{itemize}
\item \(\frac{dy^{(1)}}{dt} = \begin{pmatrix} -e^{-t} \\ e^{-t} \end{pmatrix}\), \(A y^{(1)} = \begin{pmatrix} 0 & 1 \\ -1 & -2 \end{pmatrix} \begin{pmatrix} e^{-t} \\ -e^{-t} \end{pmatrix} = \begin{pmatrix} -e^{-t} \\ -e^{-t} + 2e^{-t} \end{pmatrix} = \begin{pmatrix} -e^{-t} \\ e^{-t} \end{pmatrix} \ \checkmark\)
\item Similarly, we observe that \(y^{(2)}\) is a solution as well.
\end{itemize}

Let's finally prove that \(y^{(1)}\) and \(y^{(2)}\) are linearly independent.
To show this, it's enough to prove that \(y^{(1)}(t_0)\) and \(y^{(2)}(t_0)\) are linear independent vectors for some \(t_0\). As
\[y^{(1)}(0) = \begin{pmatrix} 1 \\ -1 \end{pmatrix} \text{ and } y^{(2)}(0) = \begin{pmatrix} 0 \\ 1 \end{pmatrix}\]
are linearly independent, we observe that \(y^{(1)}\) and \(y^{(2)}\) are linear independent functions.

Therefore, \(\{y^{(1)}, y^{(2)}\}\) is a basis.

\item
The general solution to the system is
\[y(t) = c_1 y^{(1)}(t) + c_2 y^{(2)}(t), \ t \in \mathbb{R}, c_1, c_2 \in \mathbb{R}\]

Also, \(y(0) = c_1 y^{(1)}(0) + c_2 y^{(2)}(0) = c_1 \begin{pmatrix} 1 \\ -1 \end{pmatrix} + c_2 \begin{pmatrix} 0 \\ 1 \end{pmatrix} = \begin{pmatrix} 1 \\ 1 \end{pmatrix}\).

Then, \(\begin{cases} c_1 = 1 \\ -c_1 + c_2 = 1 \implies c_2 = 2 \end{cases}\).
{\color{blue}The solution is
\[y(t) = y^{(1)}(t) + 2y^{(2)}(t), \ t \in \mathbb{R}\]
i.e., \(y_1(t) = e^{-t} + 2te^{-t}\) and \(y_2(t) = -e^{-t} + 2(1-t)e^{-t}, \ t \in \mathbb{R}\).}
\end{enumerate}
\end{proof}

\begin{xca}
Consider the system
\[\begin{cases}
\frac{dy_1}{dt} = y_2 + t \\
\frac{dy_2}{dt} = -y_1 - 2y_2 + 2t
\end{cases}\]

Find the general solution.
\end{xca}
\begin{proof}[Solution]
The homogeneous system associated to the given system is the one from the previous exercise (we know the general solution \(y_H\)). As \(b(t) = \begin{pmatrix} t \\ 2t \end{pmatrix}\) is continuous in \(\mathbb{R}\), we know that the general solution is \(y = y_H + z\) where \(z\) is a particular solution to the non-hom. system.

Let's find the particular solution \(z\).

\paragraph*{Method 1 (Method of variation of parameters)}
\(z(t) = C_1(t) y^{(1)}(t) + C_2(t) y^{(2)}(t), \ t \in \mathbb{R}\)
is a particular solution if \(C(t) = \begin{pmatrix} C_1(t) \\ C_2(t) \end{pmatrix}\) solves \(Q(t) \frac{dC(t)}{dt} = b(t), \ t \in \mathbb{R}\)
where \(Q\) is the fundamental matrix related to the basis \(\{y^{(1)}, y^{(2)}\}\).

We have that \(Q(t) = \begin{pmatrix} e^{-t} & te^{-t} \\ -e^{-t} & (1-t)e^{-t} \end{pmatrix}, \ t \in \mathbb{R}\). Recall that \(\det(Q(t)) \neq 0\).
Then \[\frac{dC(t)}{dt} = (Q(t))^{-1}b(t) = e^t \begin{pmatrix} 1-t & -t \\ 1 & 1 \end{pmatrix} \begin{pmatrix} t \\ 2t \end{pmatrix} = e^t \begin{pmatrix} t - t^2 - 2t^2 \\ t + 2t \end{pmatrix} = te^t \begin{pmatrix} 1-3t \\ 3 \end{pmatrix}\]
(exercise)

Integration yields
\begin{align*}
c_1(t) &= \int t(1-3t)e^t dt = (-3t^2 + 7t - 7)e^t \\
c_2(t) &= \int 3te^t dt = 3(t-1)e^t
\end{align*}

{\color{blue}Then, the general solution to the non-hom. system is
\[y(t) = \underbrace{c_1 y^{(1)}(t) + c_2 y^{(2)}(t)}_{y_H(t)} + \underbrace{(-3t^2 + 7t - 7)e^t y^{(1)}(t) + 3(t-1)e^t y^{(2)}(t)}_{z}\]
for \(t \in \mathbb{R}\).}

\paragraph{Method 2}
As the system is linear and of first-order, with constant coefficients and a linear polynomials as non homogeneous terms in each equation, it seems natural to look for a solution in the form \(y_1(t) = a_1 t + b_1, \ y_2(t) = a_2 t + b_2\) and look for appropriate choice of \(a_1, a_2, b_1, b_2\).

To have a solution we must have \(\begin{cases} 0 = a_2 t + b_2 + t - a_1 \\ 0 = -a_1 t - b_1 - 2a_2 t - 2b_2 + 2t - a_2 \end{cases}\) for every \(t\).

This is true if
\[\begin{cases}
a_2 + 1 = 0 \\
b_2 - a_1 = 0 \\
-a_1 - 2a_2 + 2 = 0 \\
-b_1 - 2b_2 - a_2 = 0
\end{cases}\]

The solution is \(a_1=4, a_2=-1, b_1=-7, b_2=4\).
\end{proof}