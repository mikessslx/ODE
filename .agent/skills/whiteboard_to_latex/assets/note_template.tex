% ============================================================================
%  NOTE TEMPLATE - Standard Format for N/NX.tex Files
%  Replace {X} with the note number and {MM.DD} with the date.
%  Delete all placeholder comments (% TODO:) after use.
% ============================================================================

\section{Notes {X} - {MM.DD}}

% ----------------------------------------------------------------------------
%  SUBSECTION: Main Topic
%  Use \subsection for each major topic covered in this note.
% ----------------------------------------------------------------------------
\subsection{Topic Title}

% ============================================================================
%  THEOREM STYLE (plain) - Primary Mathematical Statements
%  Body text is italicized.
% ============================================================================

% --- Theorem ---
% Use for primary mathematical statements that require proof.
\begin{theorem}[Theorem Name]
% TODO: State the theorem here.
For all \(x \in X\), the following holds:
\[\text{Mathematical expression}.\]
\end{theorem}

% --- Lemma ---
% Use for auxiliary results that support the main theorem.
\begin{lemma}
% TODO: State the lemma.
\end{lemma}

% --- Proposition ---
% Use for results that are important but less central than theorems.
\begin{proposition}
% TODO: State the proposition.
\end{proposition}

% --- Corollary ---
% Use for direct consequences of a theorem.
\begin{corollary}
% TODO: State the corollary.
\end{corollary}

% ============================================================================
%  DEFINITION STYLE - Definitions and Examples
%  Body text is roman (upright).
% ============================================================================

% --- Definition ---
% Use for introducing new concepts, terms, or mathematical objects.
\begin{definition}[Definition Name]
% TODO: Write the definition here.
% - Start text immediately after \begin{definition}, no indentation.
% - Use \textit{} for terms being defined.
% - Use enumerate/itemize for lists, flush left (no indentation).
\begin{enumerate}
\item First property or condition.
\item Second property or condition.
\end{enumerate}
\end{definition}

% --- Example ---
% Use for illustrative examples. Often grouped with \leavevmode for lists.
\begin{example}[Example Topic]
\leavevmode
\begin{enumerate}
\item[\bfseries Ex 1.]
First example description and formula:
\[d(x, y) = |x - y|\]

\item[\bfseries Ex 2.]
Second example description.
\end{enumerate}
\end{example}

% --- Exercise ---
% Use for exercises or problems for the reader.
\begin{xca}
% TODO: State the exercise.
\end{xca}

% ============================================================================
%  REMARK STYLE - Commentary and Notes
%  Body text is roman (upright), unnumbered.
% ============================================================================

% --- Remark ---
% Use for commentary, observations, or clarifications.
\begin{remark}
% TODO: Write a remark.
\end{remark}

% --- Note ---
% Use for additional notes or side comments.
\begin{note}
% TODO: Write a note.
\end{note}

% --- Recap ---
% Use at the beginning of a note to summarize previous material.
\begin{recap}[Previous Topic]
% TODO: Summarize key points from the previous lecture.
\begin{enumerate}
\item First key point.
\item Second key point.
\end{enumerate}
\end{recap}

% --- Hint ---
% Use for hints in exercises or problems.
\begin{hint}
% TODO: Write a hint.
\end{hint}

% ============================================================================
%  PROOF HANDLING
%  If proofs are hidden, use the Remark + commented proof pattern.
% ============================================================================

\begin{remark}
Proof is commented out!
\end{remark}

% \begin{proof}
% TODO: Write proof steps here.
% \end{proof}

% ============================================================================
%  END OF TEMPLATE
% ============================================================================
